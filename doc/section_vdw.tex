\section{$C_6$/$R^6$ corrections for long-range van der Waals interactions}
The correction improves the description of van der Waals (vdW) interactions in DFT.
It is based on the leading-order $C_6$/$R^6$ term for the interaction energy
between two atoms. Both energy and analytic forces are implemented.

Two flavors of the correction are implemented:       

(1) The $C_6$ coefficients and vdW radii are obtained directly
from Hirshfeld partitioning of the DFT electron density. This scheme
only requires a single damping parameter, which is fitted to binding
energies of small organic molecules and hardwired in the code for 
PBE, PBE0, revPBE, AM05, BLYP and B3LYP functionals. For more information
and citation see Ref.~\cite{TS-vdw}. Both cluster and periodic cases are implemented.

(2) The empirical $C_6$ coefficients and vdW radii must be 
specified directly. This scheme is coded for maintaining
compatibility with empirical $C_6$ approaches. In actual
applications, the usage of scheme (1) is advised, since
it is significantly more accurate and less empirical.   

\newpage

\subsection*{Tags for general section of \texttt{control.in}:}

\keydefinition{vdw\_convergence\_threshold}{control.in}
{
  \noindent
  Usage: \keyword{vdw\_convergence\_threshold} \texttt{value}\\[1.0ex]  
  Purpose: When using the vdW correction based on Hirshfeld partitioning
  of the electron density (as described in Tkatchenko and Scheffler
  2009, Ref. \cite{TS-vdw}) in a periodic system, this sets the 
  energy convergence threshold for the supercell sum over the 
  TS components. \\ 
  \texttt{value}: A small positive number (in eV). Default: 
    For unit cells with less than 100 atoms: 10$^{-6}$ eV. For structures
    with unit cell sizes above 100 atoms, the default is adjusted to
    $n_\text{atoms}\cdot10^{-8}$ eV. \\[1.0ex] 
}
Note that the vdw part of the forces may be separately converged 
to (if set) \texttt{sc\_accuracy\_forces}. 

\keydefinition{vdw\_correction\_hirshfeld}{control.in}
{
  \noindent
  Usage: \keyword{vdw\_correction\_hirshfeld} \\[1.0ex]  
  Purpose: Enables the vdW correction based on Hirshfeld partitioning
  of the electron density (as described in Tkatchenko and Scheffler
  2009, Ref. \cite{TS-vdw}). If this keyword is set in a periodic
  calculation, the sum over atom pairs is done over successively
  larger supercells, until the energy is converged to the level set by
  \keyword{sc\_accuracy\_etot} or \keyword{vdw\_convergence\_threshold} 
  and the forces (if requested) are 
  converged to within \keyword{sc\_accuracy\_forces}. \\ 
  No other
  input required. \\[1.0ex] 
}
This method is commonly referred to as the Tkatchenko-Scheffler
method. The procedure is as follows. \emph{First}, the normal
self-consistency cycle is completed for a semilocal or hybrid density
functional, most commonly PBE or PBE0. \emph{Second}, the resulting
self-consistent electron density is used to create interatomic
(pairwise) $C_6$  
coefficients. A simple pairwise van der Waals term is then added once
to the self-consistent total energy from the preceding semilocal or
hybrid functional. In other words, the Tkatchenko-Scheffler method is
normally employed as a post-processing term in a non-self-consistent
way, not during the self-consistency cycle. Since it needs to be
combined with a different density functional, you would normally use
it like this (example for ``PBE+vdW''):
\begin{verbatim}
  xc pbe
  vdw_correction_hirshfeld
\end{verbatim}
Three more caveats: (1) Do not use this method together with the
local-density approximation (LDA) unless you know exactly what you are
doing. The LDA already contains a spurious interaction term that will
lead to very strange results if added to a pairwise van der Waals
term. (2) Do not simply apply this method to a metallic system unless
you know what you are doing. (3) This is also not the (very different)
functional commonly known as the Langreth-Lundqvist or vdw-DF
functional.~\cite{Dion04} FHI-aims contains at least two
implementations of vdw-DF for those who are interested, but either
implementation is much slower than the Tkatchenko-Scheffler
pairwise interatomic sum. 

\keydefinition{vdw\_correction\_hirshfeld\_sc}{control.in}
{	
  \noindent
  Usage: \keyword{vdw\_correction\_hirshfeld\_sc} \\[1.0ex]
  Purpose: Enables the self-consistent version of the vdW correction based on Hirshfeld partitioning
  of the electron density (see Tkatchenko and Scheffler
  2009, Ref. \cite{TS-vdw}). In a periodic calculation, the energy is converged with the same criteria of the \textit{a posteriori} approach: \keyword{vdw\_correction\_hirshfeld}. \\[1.0ex]
}
This flag adds the Tkatchenko-Scheffler vdW functional as a part of the given exchange-correlation (XC) functional.
In a self-consistent scheme, the contribution of the vdW potential, $v_{\rm vdW}[n({\bf r})]=\delta E_{\rm vdW}[n({\bf r})]/\delta n({\bf r})$, is added to the XC potential to form the total effective potential in the Kohn-Sham equations.
As a result, the van der Waals interatomic contributions affect the total electron density and are computed at each self-consistent cycle, until convergence is reached.
In this way, it is possible to evaluate the effects of vdW interactions on the electron density and electronic properties, going beyond the vdW \textit{a posteriori} correction of the total DFT energy.

Note: Do not use this self-consistent flag during a relaxation. The self-consistent forces are not implemented (yet) for the Tkatchenko-Scheffler vdW functional and this will lead to inconsistency errors.

\keydefinition{vdw\_correction}{control.in}
{
  \noindent
  Usage: \keyword{vdw\_correction} \\[1.0ex]
  Purpose: Enables the empirical $C_6$/$R^6$ correction with the $C_6$ coefficients
and vdW radii specified by the user. \\[1.0ex]
}

The user needs to specify the interaction parameters for \emph{all} atomic pairs in the system
(i.e. for CNOH, there are 10 atomic pairs). This is done by putting ``vdw\_pairs $N$'', where
$N$ is the number of pairs. This should be followed by $N$ lines 
of ``vdw\_coeff atom$_i$ atom$_j$ $C_{6ij}$ $R^0_{ij}$ $d$'', where 
$C_{6ij}$ is the $C_6$ coefficient for the interaction between atom$_i$ and atom$_j$,
$R^0_{ij}$ is the corresponding vdW radius and $d$ is the damping function parameter.
A choice $d$=20 is suggested for all atomic pairs. An example for C-C interaction
is: ``vdw\_coeff C C 30.00 5.59 20.0''.

\keydefinition{vdw\_pair\_ignore}{control.in}
{
  \noindent
  Usage: \keyword{vdw\_pair\_ignore} \option{species1} \option{species2} \\[1.0ex]
  Purpose: excludes the interaction between \option{species1} and \option{species2} from any C6-correction, eg. such that metallic slabs 
  are not affected internally by introducing C6-interactions. \\[1.0ex]
}


\newpage

\subsection*{Subtags for \emph{species} tag in \texttt{control.in}:}

\subkeydefinition{species}{hirshfeld\_param}{control.in}
{
  \noindent
  Usage: \subkeyword{species}{hirshfeld\_param} \option{C6} \option{alpha} \option{R0} \\[1.0ex]
  Default: the values outlined in Ref.~\cite{TS-vdw} \\
  Purpose: To explicitly allow setting the parameters for the Tkatchenko-Scheffler van der Waals correction.
}

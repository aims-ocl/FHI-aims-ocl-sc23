\section{Energy derivatives (forces, stress) and geometry optimization}

With self-consistent Kohn-Sham orbitals, densities and total energies
available, one of the primary tasks of electronic structure theory is
to obtain energy \emph{derivatives} with respect to the nuclear
coordinates $\boldR_\text{at}$. First derivatives (``forces'') allow
to find the optimum structure and molecular dynamics on the
Born-Oppenheimer surface. Based on the structure optimum, second
derivatives then enable the calculation of the (Born-Oppenheimer)
zero-point vibrational energy of the nuclei, and of vibrational
excitations (phonons).

The present section deals with the options available in FHI-aims for
the computation of forces in FHI-aims, and with algorithms related to
geometry optimization. Specifics regarding first-principles molecular
dynamics are given in Sec. \ref{Sec:MD}, and the computation of second
energy derivatives (vibrational frequencies, zero-point energies, and
oscillator strengths) by a finite-difference technique is covered in 
Sec. \ref{Sec:vib}. 

Before coming to the full set of related keywords, 
we give some basic comments below. Please also consider using (and 
adding to) the wiki, especially if you encounter any kind of behavior
that is obviously not intended. We would like to know about such
cases. 

\textbf{One ``most important'' rule:}

For efficient local structure optimizations, do not simply use
``tight'' settings from an arbitrary starting geometry and wait.

It is usually \emph{much} more effective to use a two-step
approach. First, use ``light'' settings in a pre-relaxation
step to get the rough geometry right quickly. Then, use
``intermediate'' or ``tight'' settings for post-relaxation or
post-processing only, e.g., based on the ``geometry.in.next\_step''
file that is written out by the code by default (see below).

\textbf{Basic handling:}

For most applications, reliable structure optimizations of
atomic coordinates into a local minumum of the potential energy
surface can be obtained by simply setting the keyword \\
\keyword{relax\_geometry} \texttt{bfgs} \option{threshold} \\
\option{threshold} denotes the minimum absolute force component in
eV/{\AA} acting on any atom. Typically, a \option{threshold} value of
5d-3, i.e., 5$\cdot$10$^{-3}$ eV/{\AA}, corresponds to a very tightly 
converged structure optimization. Do not use much lower values since
you might spend substantial amounts of CPU time for no measurable gain.

For unit cell optimizations in periodic systems, the keyword
\keyword{relax\_unit\_cell} must \underline{additionally} be invoked
(otherwise, the unit cell shape will remain fixed).

Finally, individual coordinates or degrees of freedom can be fixed by
using the keyword \keyword{constrain\_relaxation}. 

Another method to run a constrained relaxation is the combined use of the keywords
\keyword{symmetry\_n\_params}, \keyword{symmetry\_params}, \keyword{symmetry\_lv}
and \keyword{symmetry\_frac}. When this block is added to a \texttt{geometry.in} and 
\keyword{relax\_geometry} (and optionally \keyword{relax\_unit\_cell}) are set in \texttt{control.in}, a geometry
relaxation is started that is constrained to a parameter reduced space defined by the user.
This can be used to enforce a symmetry-preserving relaxation by providing the exact prototype
of the given structure in its analytic form. It furthermore allows for the introduction of local
symmetries or local symmetry breaking like distortions.
{\bf Important:} The relationship between the full coordinates and the reduced parameters has to be 
linear, i.e. there is a transformation matrix A and a transformation matrix B that can map the
flattened 1-dimensional representation of the lattice cell/atomic positions to the parameter-reduced space.
For the monoclinic and triclinic lattice systems, e.g. do not use $c\cdot \cos(\beta)$ and $c\cdot \sin(\beta)$
but replace them with independent variables e.g. $d$ and $e$. {\em It is on the to-do-list to make aims 
recognize these cases internally.}\\
{\bf Hint:} You can create \texttt{geometry.in}'s already containing the required input block for a symmetry-constrained relaxation using AFLOW as of version 3.1.204. It gives access to hundreds of structure prototypes in the AFLOW Prototype Library \cite{Mehl2017,Hicks2018} throughout all spacegroups. Simply add \texttt{-{}-add\_equations} to the command. 
Example: \\
\texttt{aflow -{}-proto=AB\_cF8\_225\_a\_b -{}-params=5.64 -{}-aims -{}-add\_equations}

To obtain gradients for a given structure, but without any subsequent
structure optimization, molecular dynamics run, etc., employ the keywords
\keyword{compute\_forces}, \keyword{compute\_analytical\_stress} or 
\keyword{compute\_numerical\_stress} instead.

\keyword{relax\_geometry} \texttt{bfgs}
\option{value} calls a trust radius enhanced variant of BFGS, also
available as \texttt{trm}. This variant covers all use cases.

The \texttt{bfgs} algorithm can be tweaked by specifying an initial
guess for the Hessian matrix, which can influence the performance. The
default is the model Hessian matrix by Lindh and
coworkers,\cite{Lin95} which leads to reliable and fast optimizations in
the vast majority of scenarios.

If, for any reason, the BFGS algorithm
does not perform well for a given structure initially, a number of
options are available. The standard first step should always be to
simply restart the optimization from the updated geometry and Hessian
matrix in \texttt{geometry.in.next\_step} (see below). Alternatively,
one may preset a simple, diagonal Hessian using the keyword
\keyword{init\_hess} \option{diag} \texttt{value} (see the actual
keyword description for more details). Finally, the
\keyword{energy\_tolerance} and \keyword{harmonic\_length\_scale}
keywords decide when the \texttt{bfgs} algorithm will abort a
relaxation because a presumed deviation between the predicted and the
real energy landscape. They can be set to more forgiving values if
needed. 

For the \keyword{relax\_geometry} \texttt{bfgs} algorithm, the
handling of \emph{restarting} relaxations is formalized by writing out
a file \texttt{geometry.in.next\_step} after each relaxation
step. By default, this file contains 
both the geometry and the current estimate of the Hessian matrix of
the system that are used by the BFGS algorithm. For an organized
restart, simply copy this file to \texttt{geometry.in}, and the stored
geometry and Hessian will be used to reinitialize the BFGS algorithm
if the \texttt{control.in} file requests a structure relaxation. If the
keyword \keyword{distributed\_hessian} is in use, the current estimate of the
Hessian matrix will be stored in a separate binary file, which should not be
modified by hand.

Use the 'get\_relaxation\_info.pl' utility script to monitor the
progress of an ongoing or finished relaxation run. This information
can be immensely helpful to make sure that you are not, e.g., spending
your time optimizing the last 10$^{-5}$~eV out of an already converged 
structure relaxation.

To stop an ongoing structure relaxation in an organized way, create
the \keyword{abort\_opt} file in the respective directory.

\textbf{Stress tensor:}

For unit cell optimization (keyword \keyword{relax\_unit\_cell}), an
analytically computed stress tensor will be used where available,
thanks to work by Viktor Atalla, Christian Carbogno, and Franz 
Knuth~\cite{Knuth2015}. For some density functionals, the analytical
stress tensor is not available. In these cases, the unit cell itself
can still be relaxed, but the stress tensor must be computed
numerically from finite differences of total energies. 

The numerical finite-difference stress tensor is always available as a
fallback.

Finally, we note that, in some cases, just optimizing the unit
cell shape blindly may not be what you want. For example, in
high-symmetry structures a fit to Murnaghan's Equation of state---see
the utility provided in the \emph{utilities} directory---will be more
accurate and give you more information about the system (bulk moduli
and, in case of more than one phase, also access to transition pressures).

\textbf{Hybrid functionals:}

For hybrid functionals, analytic energy gradients and the analytic stress 
tensor are only available together with the \keyword{RI\_method} 
\texttt{LVL\_fast} version of ``resolution of identity'' of the two-electron 
Coulomb operator.\cite{Ihrig2015}

For perturbative methods (MP2 perturbation theory, RPA and beyond),
analytical gradients are not yet available. 

%\newpage

\subsection*{Tags for \texttt{geometry.in}:}

\keydefinition{constrain\_relaxation}{geometry.in}
{
  \noindent 
  Usage: \keyword{constrain\_relaxation} \option{constraint} \\[1.0ex]
  Purpose: In \texttt{geometry.in}, fixes the position of the last
    specified \keyword{atom}/\keyword{lattice\_vector} in a structure optimization. \\[1.0ex]
  \option{constraint} is a string, indicating what exactly will be
    constrained. Default: \texttt{.false.} \\
}
Allows to relax only parts of a structure, while keeping the rest at
fixed positions. Currently, the following simple options for
  \texttt{constraint} are possible:  
\begin{itemize}
  \item \texttt{.true.}: All coordinates for this atom will be constrained. 
  \item \texttt{.false.}: The relaxation of this atom will not be constrained.
  \item \texttt{x}: The $x$ coordinate of this atom is not allowed to move.
  \item \texttt{y}: The $y$ coordinate of this atom is not allowed to move.
  \item \texttt{z}: The $z$ coordinate of this atom is not allowed to move.
\end{itemize}
\emph{Attention}: If you wish to constrain more than one coordinate, the required
constraints must be specified as separate lines, like this:
\begin{verbatim}
  atom 0. 0. 0. Fe
    constrain_relaxation x
    constrain_relaxation y
\end{verbatim}
In contrast, specifying two constraints in one line will \emph{not} work. The
second constraint would simply be ignored!

\keydefinition{hessian\_block}{geometry.in}
{
  \noindent 
  Usage: \keyword{hessian\_block} \option{i\_atom} \option{j\_atom} \option{block}\\[1.0ex]
  Purpose: In \texttt{geometry.in}, allows to specify a Hessian matrix
  explicitly, with one line for each 3$\times3$ block.  The option
  \option{block} consists of 9 numbers in column-first (Fortran) order.  The
  3$\times$3 block corresponding to \option{j\_atom}, \option{i\_atom} is
  initialized by the transposed of \option{block}. The Hessian matrix is input
  in units of eV/\AA$^2$.\\
} 

If at least one \keyword{hessian\_block} line is found in the file, the
Hessian is constructed using this mechanism.  So far there is no safe-guard
from overriding Hessian blocks with subsequent lines with equal
\option{i\_atom}, \option{j\_atom}.

There are two scripts in the \texttt{utilities} directory to automatically
construct such Hessian matrix approximations.  First,
\texttt{conversions/thess2aims.py} converts a Tinker generated Hessian matrix.
Second, \texttt{Lindh.py} constructs a general purpose model matrix
\cite{Lin95}.  Please note that the Lindh model matrix is now also directly
available with \keyword{init\_hess} \texttt{Lindh}.

\keydefinition{hessian\_block\_lv}{geometry.in}
{
  \noindent 
  Usage / purpose: Like \keyword{hessian\_block}, but for Hessian
  matrix elements between lattice vector degrees of freedom.\\
} 

\keydefinition{hessian\_block\_lv\_atom}{geometry.in}
{
  \noindent 
  Usage / purpose: Like \keyword{hessian\_block}, but for Hessian
  matrix elements between lattice vector degrees of freedom.\\
} 

\keydefinition{hessian\_file}{geometry.in}
{
  \noindent
  Usage: \keyword{hessian\_file}\\[1.0ex]
  Purpose: In \texttt{geometry.in}, this keyword indicates that there exists
  a \texttt{hessian.aims} file to be used to construct the Hessian.\\
}

If \keyword{hessian\_file} is found in \texttt{geometry.in}, the Hessian is
constructed using the data in \texttt{hessian.aims}, which is a binary file
generated by geometry optimization calculations with FHI-aims (please see
\keyword{distributed\_hessian}).  The user should not try to create such a
file by hand.

\keydefinition{trust\_radius}{geometry.in}
{
  \noindent 
  Usage: \keyword{trust\_radius} \option{value} \\[1.0ex]
  Purpose: In \texttt{geometry.in}, allows to specify the initial
  trust radius value for the \texttt{trm} relaxation algorithm. \\ 
} 

This keyword is a significant exception -- it is the only
algorithmic keyword found in the \texttt{geometry.in}
file. Conceptually, it would belong into \texttt{control.in}, but
since it is written out as part of the \texttt{geometry.in.next\_step}
file which can be used to restart a structure relaxation, we keep it
here.

The keyword specifies the initial value of the ``trust radius'' used
to limit structure relaxation steps determined by the \texttt{bfgs}
(synonymous with \texttt{trm}) relaxation algorithm.

The default is set to 0.2~{\AA}. It will be overridden by the default
value of \keyword{max\_atomic\_move}, which can be set in
\texttt{control.in}. 

\keydefinition{symmetry\_n\_params}{geometry.in}
{
  \noindent
  Usage:\keyword{symmetry\_n\_params} \option{n\_total} \option{n\_lv} \option{n\_coords} \\[1.0ex]
  Purpose: In \texttt{geometry.in}, specifies the number of parameters to be optimized in a 
  symmetry-constrained relaxation. n\_total  is the total number of parameters, n\_lv is the 
  number of parameters that define the lattice cell,  n\_coords is the number of parameters defining the
  fractional atomic positions. \\
   n\_lv + n\_coords = n\_total \\
  Example for an orthorhombic cell and no parameters for the atomic positions:\\
  \texttt{symmetry\_n\_params  3  3  0}
}

This keyword also serves as a flag, setting \texttt{use\_symm\_const\_geo\_relaxation} internally to \texttt{True}.
If \texttt{n\_lv} is 0, then no unit cell relaxation is done. For unit cell relaxations make sure to set 
\keyword{relax\_unit\_cell} to \texttt{full}.

\keydefinition{symmetry\_params}{geometry.in}
{
  \noindent
  Usage:\keyword{symmetry\_params} \option{[list of variables for parameters]} \\[1.0ex]
  Purpose: In \texttt{geometry.in}, list all variables that are used as parameters separated by blanks.
  Always list the lattice parameters first. The number of variables used for parameters has to be equal
  to n\_total specifies in \keyword{symmetry\_n\_params}.\\
  Example for an orthorhombic cell and no parameters for the atomic positions:\\
  \texttt{symmetry\_params  a  b  c} 
}

\keydefinition{symmetry\_lv}{geometry.in}
{
  \noindent
  Usage:\keyword{symmetry\_lv} \option{x} , \option{y} , \option{z}\\[1.0ex]
  Purpose: Specifies the analytic form of the lattice vectors. Use exactly the same cell as given in
   \keyword{lattice\_vector} (same order of lattice vectors, same orientation), and simply replace entries
   that are free to relax by their analytic expression in terms of parameters specified by \keyword{symmetry\_params}.
   Example for an orthorhombic cell:\\
   \texttt{symmetry\_lv  a , 0 , 0}\\
   \texttt{symmetry\_lv  0 , b , 0}\\
   \texttt{symmetry\_lv  0 , 0 , c}
}

Note the comma between the components of the lattice vectors.

\keydefinition{symmetry\_frac}{geometry.in}
{
  \noindent
  Usage:\keyword{symmetry\_frac} \option{$n_1$} , \option{$n_2$} , \option{$n_3$}\\[1.0ex]
  Purpose: Specifies the analytic form of the {\em fractional} atomic positions. Use exactly the same form as given in
   \keyword{atom\_frac} (same order atoms), and simply replace entries  that are free to relax 
   by their analytic expression in terms of parameters specified by \keyword{symmetry\_params}.\\
   Example:\\
   \texttt{symmetry\_frac   0.0    ,   0.0   ,  x1}\\
   \texttt{symmetry\_frac  1./2 , 1./2 , x1 + 1./2}
}

Note the comma between the fractional coordinates of the atoms and that no species entry is needed.

\newpage

\subsection*{Tags for general section of \texttt{control.in}:}

\keydefinition{aggregated\_energy\_tolerance}{control.in}
{
  \noindent
  Usage: \keyword{aggregated\_energy\_tolerance} \option{tolerance} \\[1.0ex]
  Purpose: Sets the energy amount by which the energy across an entire
  relaxation trajectory may ever go uphill, based on the lowest known energy so far. \\[1.0ex]
  \option{tolerance} is a positive real number, in eV. Default:
    5$\cdot$10$^{-3}$~eV. \\
}
Small uphill steps of a relaxation trajectory are allowed up to the keyword
\keyword{energy\_tolerance}, but a relaxation trajectory should never go uphill
for an extended number of steps in small uphill increments. The keyword
\keyword{aggregated\_energy\_tolerance} sets an overall cap for any accepted 
uphill steps across an entire relaxation trajectory. The default is much larger
than the allowed \keyword{energy\_tolerance} in a single step and should, in
principle, never be breached. If the
\keyword{aggregated\_energy\_tolerance} criterion triggers, please
contact us.  

\keydefinition{bfgs\_extrapolation\_cap}{control.in}
{
  \noindent
  Usage: \keyword{bfgs\_extrapolation\_cap} \option{value} \\[1.0ex]
  Purpose: If the \texttt{bfgs\_textbook} algorithm of \keyword{relax\_geometry}
    deviated from a line step of 1.0, sets a maximum for the line step
    during relaxation. \\[1.0ex] 
  \option{value} is a real number $\ge$1.0 (dimensionless), the
    maximum allowed step when a variable line step used with
    BFGS. Default: 4.0 \\
}
\option{value} is dimensionless, given in units specified by the BFGS
Hessian matrix. \option{value}=1.0 corresponds to the default
extrapolation of BFGS, assuming a perfectly quadratic energy surface.


\keydefinition{bfgs\_step}{control.in}
{  

  \noindent
  Usage: \keyword{bfgs\_step} \option{type} [\option{factor} \option{cap}] \\[1.0ex]
  Purpose: Can specify a BFGS extrapolation with an adaptive line
    step instead of the default linestep for purely quadratic
    extrapolation. \\[1.0ex]
  \option{type} is a string, either \texttt{quadratic} or
  \texttt{mix}. Default: \texttt{quadratic}. \\
  \option{factor} : If \option{type} is not \texttt{quadratic},
    specifies a mixing factor to interpolate between the line step
    actually used in the last relaxation step, and the \emph{a
    posteriori} estimated optimum line step. Default: 0.1 \\
  \option{cap} : If \option{type} is not \texttt{quadratic},
    specifies an upper limit for the allowed line step. \\
}

\keydefinition{calc\_analytical\_stress\_symmetrized}{control.in}
{
  \noindent
  Usage: \keyword{calc\_analytical\_stress\_symmetrized} \option{flag} \\[1.0ex]
  Purpose: If \texttt{.false.}, calculates all 9 components of the analytical
  stress tensor. If \texttt{.true.} calculates only the upper triangle (6
  components) of the tensor and copies the result to the lower triangle. \\[1.0ex] 
  \option{flag} is a logical string, either \texttt{.true.} or
    \texttt{.false.} Default: \texttt{.true.}\\
}
Generally, it is sufficient to calculate the upper triangle of the
tensor. This flag is mainly for debugging purposes.

\keydefinition{clean\_forces}{control.in}
{
  \noindent
  Usage: \keyword{clean\_forces} \option{type} \\[1.0ex]
  Purpose: Can remove small unitary force components (rotation and
    translation of the whole structure due to residual numerical
    noise) in relaxations. \\[1.0ex] 
  \option{type} is a string, specifying whether and how any overall
    rotations / translations are removed. \\
}
The default for \option{type} depends on the exact circumstances (see
below). The following choices exist: 
\begin{itemize}
  \item \texttt{none} : No removal of residual rotations and
    translations. This is the default if any external embedding fields
    or charges are specified in \texttt{geometry.in}.
  \item \texttt{sayvetz} : Non-periodic structures: Removal of
    rotations and translations by a formal projection
    \cite{Eckart35,Sayvetz39}. In \emph{periodic} systems,
    only translations are removed.
  \item \texttt{fixed\_plane} : \emph{experimental} Simple alternative
    algorithm by constraining three atoms into a plane (implicitly
    constraining all others). 
\end{itemize}

\keydefinition{compute\_analytical\_stress}{control.in}
{
  \noindent
  Usage: \keyword{compute\_analytical\_stress} \option{flag} \\[1.0ex]
  Purpose: If \texttt{.true.}, switches on the computation of
  the analytical stress tensor. \\[1.0ex] 
  \option{flag} is a logical string, either \texttt{.true.} or
    \texttt{.false.} \\
}
Default: \texttt{.true.} if a unit cell relaxation was requested and computation
is possible. Otherwise, \texttt{.false.}

This flag allows to request an explicit analytical stress tensor computation
for an otherwise explicit single-point calculation.

The calculation of the analytical stress is limited to LDA, GGA, Meta-GGA and 
hybrid functionals and is not possible with \keyword{load\_balancing}. The vdW
correction based on Hirshfeld partitioning (\keyword{vdw\_correction\_hirshfeld})
is included into the analytical stress tensor.

\keydefinition{compute\_forces}{control.in}
{
  \noindent
  Usage: \keyword{compute\_forces} \option{flag} \\[1.0ex]
  Purpose: If \texttt{.true.}, switches on the computation of
    forces. \\[1.0ex] 
  \option{flag} is a logical string, either \texttt{.true.} or
    \texttt{.false.} \\
} 
Default: \texttt{.true.} if a geometry optimization or
molecular dynamics run was requested, or if the 
\keyword{sc\_accuracy\_forces} convergence criterion was
set. Otherwise, \texttt{.false.} 

This flag allows to request an explicit force computation for an
otherwise explicit single-point calculation. This is necessary for use
with external tools that require forces, such as a finite-difference
calculation of vibrational frequencies (see
Sec. \ref{Sec:vib}) or a transition state search (see
Sec. \ref{appendix_NEB}). In these cases, keyword
\keyword{final\_forces\_cleaned} should also be set.

\keydefinition{compute\_numerical\_stress}{control.in}
{
  \noindent
  Usage: \keyword{compute\_numerical\_stress} \option{flag} \\[1.0ex]
  Purpose: If \texttt{.true.}, switches on the computation of
  the numerical stress tensor based on central finite differences. \\[1.0ex] 
  \option{flag} is a logical string, either \texttt{.true.} or
    \texttt{.false.} \\
}
Default: \texttt{.true.} if a unit cell relaxation was requested and the
computation of the analytical stress is not possible. Otherwise,
\texttt{.false.}

If not further specified (by  \keyword{delta\_numerical\_stress}) the default
value for the scaling factor delta is set to $10^{-4}$.

\keydefinition{delta\_numerical\_stress}{control.in}
{
  \noindent
  Usage: \keyword{delta\_numerical\_stress} \option{value}                        \\[1.0ex]
  Purpose: Specifies the scaling factor delta in the computation
           of the numerical stress tensor (\keyword{compute\_numerical\_stress}). \\[1.0ex] 
  \option{value} is a dimensionless real number $> 0$. Default: $10^{-4}$.
}

\keydefinition{distributed\_hessian}{control.in}
{
  \noindent
  Usage: \keyword{distributed\_hessian} \option{flag}\\[1.0ex]
  Purpose:  If \texttt{.false.}, each MPI task holds a complete copy of the
  Hessian matrix.  If \texttt{.true.}, the Hessian matrix is distributed
  across tasks. \\[1.0ex]
  \option{flag} is a logical string, either \texttt{.true.} or
  \texttt{.false.}.  Default: \texttt{.true.} if both MPI and ScaLAPACK are
  available, \texttt{.false.} otherwise.\\
}
This keyword is particularly useful when relaxing a large structure.  Please
note that it only works if FHI-aims is built with both MPI and ScaLAPACK.  If
\keyword{relax\_geometry} \option{bfgs\_textbook} or \keyword{init\_hess}
\option{reciprocal\_lattice} is found in \texttt{control.in}, distributed
storage of the Hessian will be automatically turned off.

\keydefinition{energy\_tolerance}{control.in}
{
  \noindent
  Usage: \keyword{energy\_tolerance} \option{tolerance} \\[1.0ex]
  Purpose: Sets the energy amount by which a relaxation step can move
    upwards and is still accepted. \\[1.0ex]
  \option{tolerance} is a small positive real number, in eV. Default:
    5$\cdot$10$^{-4}$~eV. \\
}
Small upward steps during relaxation may occur as a result of a
slightly mis-guessed \texttt{bfgs} Hessian matrix somewhere
along the path, or
as a result of some residual numerical noise that leads to a
discrepancy between energies and forces. In the present code version,
such noise is always safely below the default \option{energy\_tolerance} for
reasonable settings. However, be sure to check that
the total energy does not go up across several successive steps in
a relaxation run. For the \texttt{trm} optimizer, also see
\keyword{harmonic\_length\_scale}.

\keydefinition{external\_force}{geometry.in}
{
  \noindent
  Usage: \keyword{external\_force} \option{x} \option{y} \option{z} \\[1.0ex]
  Purpose: \emph{Experimental} -- Applies an external force to the atom previous to this keyword.\\[1.0ex]
  \option{x},\option{y},\option{z} are the force components in eV/\AA\ applied to the atom in $x$, $y$, $z$ direction.\\
}
When an external force is applied it is necessary to contstrain the relaxation of at least on other atom to avoid a constant shift of the geometry. Also the value has to be reasonably chosen. Tear apart geometries can result in very flat energy landscapes, which take a large amount of time to optimize. Typically this feature should be used in cases where a small external force, e.g. a STM-tip is applied on an atomic layer and the geomtry response of this external force is of interest.

\emph{This feature is experimental since no extensive testing was done for it.}


\keydefinition{external\_pressure}{control.in}
{
  \noindent
  Usage: \keyword{external\_pressure} \option{value} \\[1.0ex]
  Purpose: \emph{Experimental} -- Applies external pressure to the unit cell.\\[1.0ex]
  \option{value} is the pressure in eV/\AA$^3$ applied to the unit cell.\\
}
In the periodic case, it is possible to apply hydrostatic pressure to the unit cell.
To actually see the effect of the external pressure, a unit cell relaxation is required
(see \keyword{relax\_geometry} and \keyword{relax\_unit\_cell}). The crystal is then
relaxed with the external pressure added to the stress tensor.

\emph{This feature is experimental since no extensive testing was done for it.}

\keydefinition{final\_forces\_cleaned}{control.in}
{
  \noindent
  Usage: \keyword{final\_forces\_cleaned} \option{flag} \\[1.0ex]
  Purpose: Decides whether spurious unitary transformations of the
    complete system (translations and rotations) are removed before
    the final output. \\[1.0ex]
  \option{flag} is a logical string, either \texttt{.true.} or
    \texttt{.false.} Default: \texttt{.true.} \\
}
This option affects directly the long-format (15 decimal) output of
total energies and forces at the end of the s.c.f. cycle in the standard
output file. If \option{flag} is \texttt{.true.}, the final output
forces are ``cleaned'' using the \texttt{sayvetz}
  \cite{Eckart35,Sayvetz39} mechanism of keyword  
\keyword{clean\_forces} (removal of translations and rotations for
cluster geometries; only translations removed for periodic systems). 

\keyword{final\_forces\_cleaned} \texttt{.true.} should be set for use
with external tools that require forces, such as a finite-difference
calculation of vibrational frequencies (see
Sec. \ref{Sec:vib}) or a transition state search (see 
Sec. \ref{appendix_NEB}). 

\keydefinition{harmonic\_length\_scale}{control.in}
{
  \noindent
  Usage: \keyword{harmonic\_length\_scale} \option{length} \\
  Purpose: The \texttt{trm}/\texttt{bfgs} algorithm of
  \keyword{relax\_geometry} judges a step by its energy gain.  Usually, one
  simply uses the energy difference.  For very short steps and rather light
  grids, however, it turns out that the qualitiy of the energy is inferior to
  the quality of the forces.  For steps shorter than \option{length}, do not
  look at the energy but use the harmonic approximation $- \Delta
  \tilde E = (\bm X_2-\bm X_1)\cdot(\bm F_2+\bm F_1)/2$ as an estimate for the
  energy gain.  If this procedure willfully accepts
  a step which increases the energy by more than \keyword{energy\_tolerance},
  the code stops to warn the user about the inconsistency between energy
  functional and forces.
  \\
  \option{length} is a length scale in \AA. Default: \texttt{0.025} \\
}
Effectively, this flag switches from a real energy minimizer to a search for a
stable zero of the force field for short step lengths.


\keydefinition{hessian\_to\_restart\_geometry}{control.in}
{
  \noindent
  Usage: \keyword{hessian\_to\_restart\_geometry} \option{flag} \\
  Purpose: Exports the current approximation to
  the Hessian matrix to \option{geometry.in.next\_step} during a 
  relaxation restart using \keyword{hessian\_block} or
  \keyword{hessian\_file}. \\ 
  \option{flag} is a logical string, either \texttt{.true.} or
    \texttt{.false.} Default: \texttt{.true.} \\
}

Note: The \texttt{geometry.in.next\_step} file is written out by
default when the \keyword{relax\_geometry} \texttt{bfgs} algorithm
is used. For the older \texttt{bfgs\_textbook} algorithm, this file is
only written when combined with the \keyword{restart\_relaxations}
keyword. 

\keydefinition{init\_hess\_lv\_diag}{control.in}
{
  \noindent
  Usage: \keyword{init\_hess\_lv\_diag} \option{value} \\[1.0ex]
  Purpose: In a geometry relaxation with a unit cell optimization,
    allows to specify the initial Hessian matrix elements used to
    estimate relaxation steps that are associated with the lattice
    vector degrees of freedom.
  \\[1.0ex]
  \option{length} is a length scale in eV/{\AA$^2$}. Default: 25
  ev/{\AA$^2$} \\ 
}
See the \keyword{init\_hess} keyword for more information on the
initial Hessian matrix used during a structure optimization.

\keydefinition{init\_hess}{control.in}
{
  \noindent
  Usage: \keyword{init\_hess} \option{type} [\option{value}]
  \\[1.0ex]
  Purpose: Defines the initial Hessian matrix that is used by the
  structure relaxation algorithms \texttt{bfgs\_textbook} and
  \texttt{bfgs} (the latter is synonymous with \texttt{trm}) of the
  \keyword{relax\_geometry} keyword.
  \\[1.0ex]
  \option{type}: Presently supported options are \texttt{diag [value]}, \texttt{Lindh 
  [value]}.
  \\[1.0ex]
  Default: \texttt{init\_hess Lindh 2.} unless a specific Hessian is
  given in the \texttt{geometry.in} file. \\ 
}

If no explicit initial Hessian matrix is given in the geometry.in file
or specified using the older \keyword{restart\_relaxations}
infrastructure, the keyword \keyword{init\_hess} can be used to
specify the initial Hessian matrix used in a structure optimization. 

With the option \option{diag}, a diagonal initial Hessian matrix is
assumed. Then, the number given by the \option{value} option sets the
diagonal elements between all atomic coordinates directly. The default
is 25\,eV/\AA$^2$. Larger values lead to a more conservative
start, smaller values lead to more aggressive initial relaxation steps.

With \option{Lindh} the Lindh model matrix \cite{Lin95} is used to
initialize the Hessian between all \emph{atomic} coordinates (usually
a very efficient guess). For  stability reasons, \option{add-value}
(in eV/\AA$^2$, defaults to 2.0\,eV/\AA$^2$) is added to all matrix 
elements on the diagonal. The parameter \option{thres} (default: 
15.0) can be used to specify the accuracy of the Lindh matrix; only
terms estimated to be larger than $e^{-\text{\option{thres}}}$ are
taken into account.   
If you are planning a large number of complex unit cell optimizations of
a similar type, we do recommend to check whether this default value is
any good. 

If a unit cell relaxation is additionally requested, 
the Hessian for the lattice degrees of 
freedom is set to be the square of the reciprocal lattice matrix and normalized to 
25\,eV/\AA$^2$. This mimics a diagonal initial Hessian if strain coordinates for the 
representation of the lattice would be used.~\cite{Pfrommer1997}
\newline

If the initial Hessian is specified explicitly by \keyword{hessian\_block} or
\keyword{hessian\_file} in \texttt{geometry.in}, this explicit hessian overrides
any information requested by the \keyword{init\_hess} keyword.


\keydefinition{line\_search\_tol}{control.in}
{
  \noindent
  Usage: \keyword{line\_search\_tol} \option{tolerance} \\[1.0ex]
  Purpose: During a \keyword{relax\_geometry} \texttt{bfgs\_textbook} geometry
    relaxation, rejects a
    relaxation step if the \emph{a posteriori} estimate for the optimal
    relaxation step deviates by more than a given tolerance from the
    actually performed line step. \\[1.0ex]
  \option{tolerance} is a real positive relative tolerance
    (dimensionless in units of the BFGS Hessian). Default:
    3.0 \\ 
}
If keyword \keyword{relax\_geometry} \texttt{bfgs\_textbook} is used,
\keyword{line\_search\_tol} will reject a relaxation step based on an
\emph{a posteriori} estimate of the true optimum step length (see 
Ref. \cite{Blum08} for details). In short, a step of length $l$ in
dimensionless units (units relative to the BFGS Hessian matrix) is
rejected if 
\begin{equation}
  |l_\text{\emph{a posteriori}} - l_\text{actual}|/l_\text{actual} \ge
   \mathtt{tolerance} \, .
\end{equation}

\keydefinition{line\_step\_reduce}{control.in}
{
  \noindent
  Usage: \keyword{line\_step\_reduce} \option{value} \\[1.0ex]
  Purpose: Determines the behaviour of FHI-aims if a
    \keyword{relax\_geometry} \texttt{bfgs\_textbook} relaxation step was
    rejected. \\[1.0ex]
  \option{value} is either a string descriptor, or a numerical
    value. Default: \texttt{automatic} . \\
}
During a structure optimization using the \keyword{relax\_geometry}
\texttt{bfgs\_textbook} algorithm, FHI-aims checks if a performed relaxation
step actually reduced the total energy compared to the
previous step. The allowed upward tolerance is set by keyword
\keyword{energy\_tolerance}. If an upward step is encountered,
FHI-aims first attempts to repeat the step with a smaller step
length. If this smaller step also leads upwards, FHI-aims assumes
that the BFGS Hessian has exceeded its validity limits, and
reinitializes the Hessian before continuing. 

If \texttt{value} is set to \texttt{automatic}, the BFGS step is
repeated with an \emph{a posteriori} estimated length $l$ given by the
previous (failed) step (see Ref. \cite{Blum08}).

Alternatively, \texttt{value} can be set to a real positive number $<$1,
corresponding to the default factor by which the previous step length
is reduced. 

\keydefinition{max\_atomic\_move}{control.in}
{
  \noindent
  Usage: \keyword{max\_atomic\_move} \option{value} \\[1.0ex]
  Purpose: Maximum allowed step length taken during relaxation. \\[1.0ex]
  \option{value} is a real positive upper bound for the maximum allowed
    change in single atomic coordinate, in {\AA}. Default: 0.2 {\AA}. \\
}
If the \texttt{bfgs\_textbook}-predicted change in an atomic coordinate exceeds
\keyword{max\_atomic\_move}, the length of the entire step (all
coordinates) will be scaled down to not exceed the maximum allowed
displacement.

\keydefinition{max\_relaxation\_steps}{control.in}
{
  \noindent
  Usage: \keyword{max\_relaxation\_steps} \option{number} \\[1.0ex]
  Purpose: A structure optimization will be aborted after exceeding a
    prescribed maximum number of steps. \\[1.0ex]
    \option{number} is the prescribed maximum step number. Default:
    1000 . \\
}

\keydefinition{min\_line\_step}{control.in}
{
  \noindent
  Usage: \keyword{min\_line\_step} \option{value} \\[1.0ex]
  Purpose: Sets the minimum possible (dimensionless) BFGS linestep
    before the BFGS Hessian is reinitialized \\[1.0ex]
  \option{value}: A real positive number (between 0 and 1). Default:
    0.1 \\
}
See keyword \keyword{line\_step\_reduce}. If, after a rejected
\texttt{bfgs\_textbook} relaxation step, the reduced, dimensionless BFGS line
step decreases below \keyword{min\_line\_step}, the BFGS Hessian
matrix estimate is considered inappropriate for the present point in
space, and the Hessian is reinitialized.

\keydefinition{numerical\_stress\_save\_scf}{control.in}
{
  \noindent
  Usage: \keyword{numerical\_stress\_save\_scf} \option{flag} \\[1.0ex]
  Purpose: Controls if \keyword{constrain\_relaxation}
  directives are used to determine implicitly if a component of the numerical
  stress has to be calculated. This greatly accelerates unit cells
  with high symmetries (e.g. orthorhombic).  \\[1.0ex]
 \option{flag} is a logical string, either \texttt{.true.} or
    \texttt{.false.} Default: \texttt{.true.} \\
}

\keydefinition{orthonormalize\_eigenvectors}{control.in}
{
  \noindent
  Usage: \keyword{orthonormalize\_eigenvectors} \option{flag} \\[1.0ex]
  Purpose: Specifies whether or not the wave function coefficients
    from the previous geometry will be re-orthonormalized before
    initializing a new relaxation step. \\[1.0ex]
  \option{flag} is a logical string, either \texttt{.true.} or
    \texttt{.false.} Default: \texttt{.true.} \\
}
The \keyword{orthonormalize\_eigenvectors} keyword allows to
reorthnormalize the converged self-consistent Kohn-Sham orbitals
$c_{jl}$ after a relaxation step. These are then used to
reinitialize the electron density for the next relaxation step.

Due to the change in atomic positions, the wave function coefficients
$c_{jl}$ for the earlier geometry are no longer orthonormal after the
relaxation step. The consequence is an initial electron density which
no longer satisfies the correct electron count (i.e., the system may
appear to be charged immediately after a relaxation step, although a
neutral system was requested). In principle, this does not matter for
the outcome of a calculation, since the self-consistent solution will
be independent of the starting point. In many cases, there is no clear
benefit in terms of the s.c.f. convergence duration from
orthonormalizing the $c_{jl}$ prior to the reinitialization; however, some
cases with unstable s.c.f. convergence may benefit significantly.

\keydefinition{relax\_geometry}{control.in}
{
  \noindent
  Usage: relax\_geometry \option{type} \option{tolerance} \\[1.0ex]
  %
  Purpose: Specifies if a structure optimization (geometry relaxation)
    is requested, and which. \\[1.0ex]
  %
  \option{type} specifies the type of requested structure
    optimization. Default: \texttt{none}. \\
  \option{tolerance}: Specifies the maximum residual
    force component per atom (in eV/{\AA}) below which the geometry
    relaxation is considered converged. \\
}
Finds the nearest minimum of the Born-Oppenheimer potential energy
surface for the nuclei.

For periodic calculations: If you are looking to relax not just atomic
coordinates but also the unit cell shape (lattice vectors), you do
need to specify an additional keyword: \keyword{relax\_unit\_cell}.

The presently supported options for \option{type} are \texttt{none} and
\texttt{trm} (synonymous with \texttt{bfgs}), as well as \texttt{trm\_2012} and 
\texttt{bfgs\_textbook} for reference with older FHIaims versions.
\begin{itemize}
  \item \texttt{bfgs} or \texttt{trm} is the recommended default. It
    uses a trust radius method enhanced 
    version of the Broyden-Fletcher-Goldfarb-Shanno (BFGS) optimization
    algorithm (see Ref. \cite{Nocedal06-numopt}, which was the basis for
    J\"urgen Wieferink's code effort in this area). In our tests, this version
    appears to give the fastest convergence reliably.
    
    As of December 2018, it additionally implements preconditioning of the 
    lattice-lattice Hessian as 
	  explained in \keyword{init\_hess} and 
	  preserves fractional coordinates of the atomic positions when predicting 
	  a new lattice. This mimics the optimization in strain coordinates. 
	  \cite{Pfrommer1997}
	  \newline
	\item \texttt{trm\_2012} Implements the former \texttt{trm} method \emph{without} 
	effective strain coordinates.
		\newline
  \item \texttt{bfgs\_textbook} is only available as a fallback
    option. It is \textbf{not} 
    intended or recommended for routine use and it will, in general,
    perform worse than \texttt{bfgs} -- sometimes much worse. It invokes
    a Cartesian BFGS optimization algorithm that adheres strictly to
    the textbook \cite{NumericalRecipes}; we note that our coded
    version is based on the concise description found at Wikipedia.  
    See Ref. \cite{Blum08} for some additional details of our version.
\end{itemize}

A reliable force convergence criterion \option{tolerance} for most
structures is 10$^{-2}$ eV/{\AA} or 5$\cdot$10$^{-3}$ eV/{\AA}. 
\textbf{Do not use significantly smaller values unless you have a specific 
reason. Smaller values may cost much computer time for essentially no
further measurable total energy minimization.}

Going to a much smaller \texttt{tolerance} value may only be useful for
some very specific purposes, for example, high-accuracy finite difference
calculations for vibrational properties. In other scenarios, if
\texttt{tolerance} is set to a too small value by default, 80\% or
more of your CPU time may be spent groping around in the last meV of
the structure optimization.

If tighter settings of the \texttt{tolerance} parameter are used,
do not forget that tighter s.c.f. convergence accuracy settings
may also be required to get accurate gradients in the first
place. Ideally, use the \keyword{sc\_accuracy\_rho} keyword for this
purpose, not \keyword{sc\_accuracy\_forces} or
\keyword{sc\_accuracy\_stress} (see below).

In other words, use the \texttt{tolerance} criterion for a structure
relaxation run wisely -- decide what is the physical quantity you are
actually interested in, and then check which value of the
\texttt{tolerance} criterion is safe but still efficient.

The relaxation algorithm can be greatly sped up by using a somewhat
intelligent guess for the Hessian matrix used in the initial step. 
By default, FHI-aims now sets the general purpose model matrix due to Lindh
and coworkers \cite{Lin95} with a slight modification. If, for some
reason, a particular initial geometry does not appear to play well
with the \texttt{Lindh} Hessian, a simpler, slower, but more resilient
diagonal approximation to the initial Hessian matrix can also be used.
For more information
see the \keyword{init\_hess} keyword above.

The \keyword{energy\_tolerance} and
\keyword{harmonic\_length\_scale} keywords can be set to more
forgiving values if the \texttt{bfgs} algorithm decides to 
abort relaxations because of a presumed deviation between the
predicted and the real energy landscape.

\textbf{Another important warning:} Evaluating the forces and
the stress tensor is \textbf{much} more expensive than a normal
iteration during s.c.f. convergence. The current default behavior of
FHI-aims avoids any double computations of forces and stress tensors,
relying instead on a sufficiently tight convergence criterion
\keyword{sc\_accuracy\_rho} to determine s.c.f. convergence and only
then calculating forces and stresses \emph{once} per geometry step. 

While one can in principle check the s.c.f convergence of forces /
stresses explicitly, the cost of multiple evaluations
of forces / stresses for a single geometry can be \textbf{very} high.
Therefore, we recommend to \emph{never} use the keywords
\keyword{sc\_accuracy\_forces} or \keyword{sc\_accuracy\_stress} 
in a \texttt{control.in} file unless there is a specific need. Do not
set these keywords routinely. 


\keydefinition{relax\_unit\_cell}{control.in}
{
  \noindent
  Usage: \keyword{relax\_unit\_cell} \option{type}  \\[1.0ex] 
  Purpose: Relaxes unit cell (lattice vectors) using the structure 
           optimization as specified in \keyword{relax\_geometry}. \\[1.0ex]
  \option{type} specifies the type of requested unit cell 
    optimization. Presently supported options: \texttt{none},
    \texttt{full}, \texttt{fixed\_angles}. Default: \texttt{none}\\
}
Allows to optimize the lattice vectors of a periodic calculation, in
addition to the normal atomic coordinates inside the unit cell. This
keyword is not on by default, as automatically optimizing the unit
cell of (say) a surface calculation could do a lot of unintended
harm. Possible settings:
\begin{itemize}
  \item \texttt{none} : Unit cell will be kept fixed, no
    optimization. 
  \item \texttt{full} : All \keyword{lattice\_vector} degrees of
    freedom will be relaxed, except those affectd by explicit
    constraints.
  \item \texttt{fixed\_angles} : All angles between lattice vectors
    will be constrained (kept fixed), only the lengths of each lattice
    vector are varied. (This option used to be called \texttt{shape}, but
    that is a misunderstandable term. The \texttt{shape} term will be
    removed in the future to avoid confusion.)
\end{itemize}
This keyword should be used only together with \keyword{relax\_geometry}.
Individual lattice vectors or its components can be constrained by using 
\keyword{constrain\_relaxation}.

If the computation of the analytical stress is possible regarding the chosen
computational settings, the analytical stress is used for the unit cell
relaxation. Otherwise, the numerical stress is used. With
\keyword{stress\_for\_relaxation}, one can explicitly choose either numerical or
analytical stress for the unit cell relaxation.

If a unit cell relaxation produces strange results with the analytical
stress, here are some potential remedies:
\begin{enumerate}
  \item A very possible reason may be because the integration grids
    are not dense enough. This could especially well be the case for
    ``light'' settings. One remedy is to just use the integration
    grids from ``tight'' and the basis functions from ``light''. 
  \item Another possible remedy is to switch the way the integration
    weights are calculated to a slightly slower, non-default
    version. E.g., change the \keyword{partition\_type} to a spherical
    one like \texttt{rho\_r2}. 
  \item Finally, you may wish to set the convergence of the analytical
    stress with \keyword{sc\_accuracy\_stress} to an explicit, final
    value. \textbf{Only ever set this keyword for test purposes,
    though, not routinely. The cost for too many analytical stress
    evaluations can be disproportionately large.}
\end{enumerate}

\keydefinition{stress\_for\_relaxation}{control.in}
{
  \noindent
  Usage: \keyword{stress\_for\_relaxation} \option{type} \\[1.0ex]
  Purpose: Use either numerical or analytical stress for unit cell
  relaxations.\\[1.0ex] 
  \option{type} can be either \texttt{numerical} or
    \texttt{analytical}. Default: Chosen automatically based on computational
    settings. \\
}
To perform an actual unit cell relaxation, one has to set
\keyword{relax\_unit\_cell}. If one chooses \texttt{analytical} but the 
computation of the analytical stress is not possible, FHI-aims will abort.

\keydefinition{restart\_relaxations}{control.in}
{
  \noindent
  Usage: \keyword{restart\_relaxations} \option{flag}
    \\[1.0ex]
  Purpose: To save all necessary data to restart a \texttt{bfgs\_textbook}
    relaxation without needing to recompute the relaxation path in
    DFT. \\[1.0ex] 
 \option{flag} is a logical string, either \texttt{.true.} of
 \texttt{.false.} Default: \texttt{.false.} \\
}

Note: The \keyword{restart\_relaxations} keyword will work with both
the \texttt{bfgs} (\texttt{trm}) or the \texttt{bfgs\_textbook}
versions of the \keyword{relax\_geometry} keyword. However, it is only
needed for the \texttt{bfgs\_textbook} version, if a restart is
desired. For the normal (and faster) \texttt{bfgs} (\texttt{trm})
version, all that is needed is the information contained in
\texttt{geometry.in.next\_step}. 

In order to restart
from an already known estimated Hessian, rather from unity (the
default in the absence of any previous knowledge), the Hessian can be
rebuilt from a series of stored atomic cordinates and forces.  

This information is exported to a separate file
\texttt{relaxation\_restart\_file.FHIaims}, which will be read and
evaluated by a newly beginning FHI-aims calculation if the
\keyword{restart\_relaxations} keyword is set to true.

The flag also needs a \texttt{geometry.in} file that is an exact copy
of the \texttt{geometry.in.next\_step} file, which is updated during
the structure optimization.

See also \keyword{restart} for keeping the wave functions as well. 

\keydefinition{write\_restart\_geometry}{control.in}
{
  \noindent
  Usage: \keyword{write\_restart\_geometry} \option{flag} \\
  Purpose: During a structure optimization, exports the current
  geometrys and approximation to the Hessian matrix to a file 
  \option{geometry.in.next\_step}. \\ 
  \option{flag} is a logical string, either \texttt{.true.} or
    \texttt{.false.} Default: \texttt{.true.} \\
}

Note: The \texttt{geometry.in.next\_step} file is written out by
default when the \keyword{relax\_geometry} \texttt{bfgs} algorithm
is used. For the older \texttt{bfgs\_textbook} algorithm, this file is
only written when combined with the \keyword{restart\_relaxations}
keyword. 


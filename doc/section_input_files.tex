\section{The mandatory input files: \texttt{control.in} and \texttt{geometry.in}}

As discussed in Sec. \ref{Sec:running}, FHI-aims requires exactly two input
files---\texttt{control.in} and \texttt{geometry.in}---located in
the same directory from which the FHI-aims binary is invoked. To start
FHI-aims, no further input should be needed.\footnote{A few specific keywords
  (e.g., a restart of an existing calculation from an earlier wave function
  or density matrix) may require
  additional input that simply can not be included in user-edited file. Such
  input files will be described with the appropriate tasks.}

\begin{figure}
  \small
  \begin{verbatim}
# Geometry for water -- needs to be relaxed as the water molecule 
# described here has a 90degree bond angle and a 
# 1 Angstrom bond distance ... 
atom    0.00000000    0.00000000    0.00000000    O
atom    0.70700000   -0.70700000    0.00000000    H 
atom   -0.70700000   -0.70700000    0.00000000    H 
  \end{verbatim}
  \normalsize
  \caption{\label{Fig:geometry.in}
    Example input file \texttt{geometry.in}, provided with the simple
    test case (relaxation of H$_2$O) described in Sec. \ref{Sec:running}.
  }
\end{figure}

\begin{figure}
  \small
  \begin{verbatim}
#########################################################################################
#
#  Volker Blum, 2017 : Test run input file control.in for simple H2O
#
#########################################################################################
#
#  Physical model
#
  xc                 pbe
  spin               none
  relativistic       none
  charge             0.
#
#  Relaxation
#
  relax_geometry   bfgs 1.e-2
#
################################################################################
#
#  FHI-aims code project
#  VB, Fritz-Haber Institut, 2009
#
#  Suggested "light" defaults for H atom (to be pasted into control.in file)
#  Be sure to double-check any results obtained with these settings for post-processing,
#  e.g., with the "tight" defaults and larger basis sets.
#
################################################################################
  species        H
#     global species definitions
    nucleus             1
    mass                1.00794

[...]

  \end{verbatim}
  \normalsize

  \vspace*{-4.0ex}

  \caption{\label{Fig:control.in}
    Excerpts from the example input file \texttt{control.in}, provided with the simple
    test case (relaxation of H$_2$O) described in Sec. \ref{Sec:running}. A
    section of \emph{general} (system-wide) run-time settings is separate from
    individual sections that describe settings specific to certain
    \emph{species} (chemical elements).
  }
\end{figure}

Figures \ref{Fig:geometry.in} and \ref{Fig:control.in} show as examples the
\texttt{geometry.in} and \texttt{control.in} files used for the simple test
case (relaxation of a water molecule) described in Sec. \ref{Sec:running}. The
philosophy of their separation is simple:
\begin{itemize}
  \item \texttt{geometry.in} contains only information directly related to the
    atomic structure for a given calculation. This obviously includes atomic
    positions, with a description of the particulars of each element (or
    \emph{species}) expected in \texttt{control.in}. In addition, lattice
    vectors may be defined if a periodic calculation is required. Any other
    information is only given here if it is \emph{directly} tied to the atom
    in question, such as an initial charge, initial spin moment, relaxation
    constraint etc. The order of lines is irrelevant, except that information
    specific to a given atom must follow \emph{after} the line specifying that
    atom, and \emph{before} any following atom is specified.
  \item \texttt{control.in} contains all other runtime-specific
    information. Typically, this file consists of a \emph{general} part,
    where, again, the particular order of lines is unimportant. In addition,
    this file contains \emph{species} subtags that are references by
    \texttt{geometry.in}. Within the description of a given species, the order
    of lines is again unimportant, but \emph{all} information concerning the
    same species must follow the initial \emph{species} tag in one block.
\end{itemize}
In both files, the choice of units is {\AA} for length parameters, and eV for
energies; derived quantities are handled accordingly. Lines beginning with a \#
symbol are treated as comments, and empty lines are ignored. Finally, each
non-comment line has the following, free-format structure:

\begin{verbatim}
  keyword value <value> <value>
\end{verbatim}

Generally, all keywords and values are case sensitive: Do not expect FHI-aims
to understand an ``XC'' keyword if the specified syntax is ``xc''. 

It is the objective of the \emph{next} chapter, Chapter \ref{Ch:full},
to list all legitimate keywords in FHI-aims, and to describe their
function. 

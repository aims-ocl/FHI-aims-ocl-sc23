\chapter[Multiple Instances of FHI-aims]{Split large cluster allocation to multiple
aims instances}
\label{appendix_multiaims}

Specific tasks need to perform many small jobs rather than a few big ones. The
job submission of such tasks on large compute facilities, which typically require
at least a few hundreds MPI processes, can be cumbersome. FHI-aims allows for
running multiple instances of itself and split a global MPI communicator into
smaller, independent parts. This allows for a unified approach to submit one
large job and let the splitting happening in aims.

To use this feature, you need to compile the multiaims executable by compiling
the multiaims target. When using CMake, this is controled via the
\texttt{MULTIAIMS} option. When using Makefile, the target
\texttt{multi.scalapack.mpi} should be used. In either case, a C compiler is
mandatory in addition to the usual requirements.

The multiaims environment needs its jobs distributed in subdirectories labeled
1 to $N$, where $N$ is the total number of jobs. In the main directory one
additional control file \textit{multiaims.in} needs to be specified. This file
contains two keywords:

\keydefinition{tasks\_per\_subjob}{multiaims.in}
{
  \noindent
  Usage: \keyword{tasks\_per\_subjob} \texttt{value} \\[1.0ex]
  Purpose: Specifies the number of processes working on one subtasks. Should be
    a divisor of total number of processes.
}

\keydefinition{start\_id}{multiaims.in}
{
  \noindent
  Usage: \keyword{start\_id} \texttt{value} \\[1.0ex]
  Purpose: Specifies the first job folder to start with.
}


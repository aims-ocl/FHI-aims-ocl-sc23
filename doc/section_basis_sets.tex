\section{Specifying the basis (functions, empty sites, k-points, ...)}

Among the technical choices in FHI-aims, the choice of the basis set
is by far the most important one, both regarding the efficiency and
the desired accuracy of a calculation. The shape and details of the
basis sets used are thus kept as obvious as possible to the user. At
the same time, nobody should be required to type in an entire
basis set plus additional specifications from scratch just to run a
production calculation. 

As described in Ref. \cite{Blum08}, the basis functions of FHI-aims
take the format
\begin{equation}
  \phi(\boldr) = u(r)/r \cdot Y_{lm}(\theta,\phi)
\end{equation}
in spherical coordinates ($r,\theta,\phi$) with respect to a given
atomic center. Each radial function $u(r)$ is numerically tabulated on
a dense logarithmic radial grid, and evaluated as a cubic spline
function in other parts of the code. Finally, most radial function
types are subject to a cutoff potential of radial with $w$, ensuring
that $u(r)$=0 for $r>r_\text{cut}=r_\text{onset}+w$.

In periodic calculations, the full basis specification additionally
includes the $k$-point grid for Bloch functions in the first Brillouin
zone. Unlike in many other implementations, this is \emph{not} a
performance-critical setting in FHI-aims, and should be set to a well
converged value if possible.

The recommended approach to basis sets in FHI-aims is 
twofold:
\begin{itemize}
  \item \emph{First}, obtain the basic description of each required
    element by copy-pasting one of the preconstructed
    \texttt{species\_default} files into your \texttt{control.in}
    file. The preconstructed \texttt{species\_default} files address
    all standard specifications associated with a single
    \keyword{species}, including the integration grids, the Hartree
    potential, and most importantly the basis set. 
  \item \emph{Second}, \textbf{edit the copy-pasted species\_defaults
    file} to match your specific accuracy and efficiency
    requirements. For the basis set, this is done by adjusting the
    species-dependent keywords described below. Most importantly,
    \emph{complete} basis sets are listed at the end of each
    \texttt{species\_default} file. You can increase/decrease the
    basis set accuracy by successively uncommenting / commenting
    \emph{tiers} of the basis set. \texttt{Note} that each higher
    \emph{tier} must only be used if \emph{all} lower \emph{tiers} are
    active. For example, it does \emph{not} make sense to use all tier
    2 basis functions if the first tier is not used.
\end{itemize}

In addition to our own case studies,\cite{Blum08} the accuracy of the
``tier'' (sometimes called ``FHI-aims-2009''\cite{Zhang2013}) preconstructed basis 
sets for semilocal and hybrid DFT calculations was established in
several benchmark assessments.\cite{Lejaeghereaad3000,Jensen2017}

For beyond-DFT methods like MP2, $GW$, the random-phase approximation,
etc., basis set convergence is very different and absolute convergence
is often not possible. In the case of beyond-DFT methods,
we recommend to ascertain basis set convergence by performing specific convergence
tests for any important results. For total-energy differences between different
structures, a counterpoise correction can often be employed (using the
\keyword{empty} keyword to create sites which have basis functions but
no atoms). For beyond-DFT calculations for light elements (H-Ar), the
NAO-VCC-nZ basis sets\cite{Zhang2013} are additionally available among the species 
defaults and provide reliable convergence for total-energy methods
(see Ref. \cite{Zhang2013} for details). 

Finally, we can also use different approaches to create the
free-atom-like core and valence basis functions that are included in
the so-called ``minimal basis'' of the NAO basis sets provided with 
FHI-aims (unless explicitly excluded using the
\keyword{include\_min\_basis} keyword). If semilocal density
  functionals are used in the overall calculation, these atomic 
radial functions are created for the same exchange-correlation
functional as requested by the \keyword{xc} keyword of the
\texttt{control.in} file. For hybrid density functionals, FHI-aims has 
historically used LDA- or GGA-derived free-atom-like basis functions
by default (still current as of August 2017, but slated to be changed
in the future), but the definition of the ``minimal basis'' can be
changed in a limited way using the keywords
\keyword{atomic\_solver\_xc} or \keyword{atomic\_solver}. See, for
instance, Figure 7 in Ref. \cite{Ren12a} for the effect of changing
the mimimal basis definition by keyword \keyword{atomic\_solver\_xc}
on total energy convergence for a simple example.  

\newpage

\subsection*{Tags for general section of \texttt{geometry.in}:}

\keydefinition{empty}{geometry.in}
{
  \noindent
  Usage: \keyword{empty} \option{x} \option{y} \option{z}
  \option{species\_name} \\[1.0ex]
  Purpose: Specifies the initial location and type of a \emph{site}
  where \emph{only} the basis functions (but not the nucleus) of a
  given species are placed.
  \\[1.0ex]
  Restriction: Currently not functional with periodic boundary
    conditions. The use of this option should be avoided for physical
    reasons if a structure relaxation is requested.\\[1.0ex]
  \option{x}, \option{y}, \option{z} are real numbers (in \AA) which
  specify the atomic position. \\[1.0ex]
  \option{species\_name} is a string descriptor which names the element on
    this atomic position; it must match with one of the species descriptions
    given in \texttt{control.in}. \\
}
This allows to place extra basis functions at specified locations
outside the actual atoms, e.g., allowing for a counterpoise
correction of basis set superposition errors.

\newpage

\subsection*{Tags for general section of \texttt{control.in}:}

\keydefinition{atomic\_solver}{control.in}
{\noindent
  Usage: \keyword{atomic\_solver} \option{string} \\[1.0ex]
  Purpose: Changes the atomic solver library that generates the
  free-atom-like radial functions (or free-ion-like radial functions)
  used, e.g., in the minimal basis part of the NAO basis sets. 
  \\[1.0ex]
  \option{string} is the name of the solver to be used, either \texttt{sratom} or
  \texttt{atom\_sphere}. Default: \texttt{sratom}  \\
}
The definition of free atom radial functions used as the ``minimal
basis'' of FHI-aims affects the absolute convergence of total energies
calculated by FHI-aims. The radial shape of the
free-atom core and valence functions towards the nucleus is nearly
exact also for bonded structures if the same exchange-correlation
functional is used and thus, e.g., using DFT-PBE generated radial functions
when using \keyword{xc} \texttt{pbe} will improve the convergence of
total energies. (Other quantities, such as atomization energies or other
energy differences, will often exhibit better convergence than the
total energy, since the effect of the exact shape of the minimal basis
functions near the nucleus often cancels to a reasonable extent. See,
e.g., Ref. \cite{Jensen2017} for a study of these effects for
Gaussian-type and NAO basis sets compared to accurate reference values.)

Options
for \texttt{string}: 

\texttt{sratom}: FHI-aims' default solver for the electronic structure
of spherical free atoms on a dense logarithmic grid, called ``sratom'' (for
``scalar relativistic atom'') is the same solver as used in the
Fritz Haber Institute 1998 pseudopotential generation code by Martin
Fuchs and coworkers, reference \cite{FuchsFHI98PP}. It was modified by
Timo Jacob to incorporate ZORA scalar relativity when needed. This
solver can handle semilocal density functionals but not exact
exchange. Thus, for hybrid density functionals in FHI-aims, the
``minimal basis'' radial functions produced by sratom are semilocal
DFT, not hybrid functional basis functions. This leads to a slower
convergence of absolute total energies with hybrid functionals in
FHI-aims (the error in atomization energies cancels to a good
extent). See Figure 7 in Ref. \cite{Ren12a} for the magnitude of this
effect for the example of the convergence of Hartree-Fock calculations
for Au$_2$ with an LDA-generated minimal basis compared to the same
calculation, but with a minimal basis derived using the
Krieger-Li-Iafrate (KLI) approximation
\cite{Krieger1990,Krieger1992a,Krieger1992b} to the exact-exchange 
optimized effective potential. ``sratom'' is the current default
solver for radial functions in FHI-aims.

\texttt{atom\_sphere}: This solver for spherical free atoms was
developed in Stefan Goedecker's group for many years (beginning with
Ref. \cite{Goedecker1992}) and includes support for semilocal and
hybrid density functionals. In FHI-aims, ``atom\_sphere'' is used as a
library under the Lesser General Public License (LGPL). As of this writing
(August 2017), only non-relativistic calculations are supported by 
``atom\_sphere''. In this case, the resulting total energies
for hybrid density functionals converge precisely as well as their
semilocal DFT equivalents, as shown, e.g., in
Ref. \cite{Jensen2017}. ``atom\_sphere'' will only work if support 
for libxc was compiled into the FHI-aims binary used.


\keydefinition{atomic\_solver\_xc}{control.in}
{\noindent
  Usage: \keyword{atomic\_solver\_xc} \option{string} \\[1.0ex]
  Purpose: Changes the exchange-correlation functional used to generate the
  free-atom-like radial functions in the minimal basis
  part of the NAO basis sets.
  \\[1.0ex]
  Restriction: This keyword only has an effect if
  \keyword{atomic\_solver} \texttt{sratom} is used. \\[1.0ex]
  \option{string} is the name of the exchange-correlation functional
  to be used for the free-atom solution. Default: Internal defaults
  (no specific option).  \\
}
There is currently only one specific option for this keyword:

\texttt{KLI}

for the Krieger-Li-Iafrate (KLI) approximation
\cite{Krieger1990,Krieger1992a,Krieger1992b} to the exact-exchange
optimized effective potential. Figure 7 in Ref. \cite{Ren12a} is an
example of the effect of the choice of different minimal basis
functions on the convergence of the total energy of a Hartree-Fock
calculation for a heavy element. In that case, the KLI approximation
is closer to the converged result since the radial behavior of the
minimal basis functions towards the nucleus is closer to the
Hartree-Fock result, and because the radial function behavior of the
free atom near the nucleus is nearly identical to the near-nuclear
behavior of the Kohn-Sham orbitals in the bonded structures. In other
words, for Hartree-Fock, KLI represents the nuclear cusp better than
the local-density approximation (which is the present default choice
of minimal basis radial functions for Hartree-Fock). 

In all cases where \keyword{atomic\_solver\_xc} is not specified, the
choice of the exchange-correlation functional for free atoms varies
with the chosen setting for the \keyword{xc} keyword, but is not
necessarily identical. For local and semilocal density functionals,
the choice of functional for the minimal basis is generally identical
to keyword \keyword{xc}, but for hybrid functionals, the choice may
vary and is currently ``best'' documented in the source code (in
subroutine get\_free\_atoms.f90).


\keydefinition{calculate\_atom\_bsse}{control.in}
{\noindent
  Usage: \keyword{calculate\_atom\_bsse} \option{flag} \\[1.0ex]
  Purpose: Allows calculation of the basis set superposition error
  (BSSE) corrected atomization energy. 
  \\[1.0ex]
  \option{flag} is a logical string, either \texttt{.false.} or
    \texttt{.true.} Default: \texttt{.false.}  \\
}
\textit{This keyword automates a specialized version of the
  counterpoise correction and should only be used with great care. A
  general counterpoise correction for molecules can be implemented
  manually and in separate steps using the \keyword{empty}
  keyword. The atomization BSSE correction implemented above, if used, must be
  checked very carefully to ensure that all single-atom reference
  calculations carried out in the process reached the exact same
  atomic reference state. Many important atoms have more than one
  self-consistent solution, and mixing different self-consistent
  solutions may produce completely erratic results. Our recommendation
  therefore to carry out any counterpoise correction manually instead,
  by doing separate single-point FHI-aims calculations with different
  molecular fragments and different basis definitions.} 

The atomization BSSE correction for a molecular structure is defined
as: 
\begin{equation}
  \Delta_{ac}=\sum_x[E^x(x)-E^x(sys)]
\end{equation}
where $E^x(x)$ is the energy of the atom \textit{x} calculated using
only its basis set and $E^x(sys)$ is its energy calculated with the
basis set of the whole structure. The BSSE corrected total energy is
then 
\begin{equation}
  E^{BSSE}=E^{sys}(sys)+\Delta_{ac}
\end{equation}
The atomization BSSE correction is usually small in the case of the
LDA/GGA functionals, but can become significant for methods with
explicit correlation like RPA and MP2. 

The BSSE corrected atomization energy implemented here, on the other hand, is 
\begin{equation}
  E^{BSSE}_{at}=E^{sys}(sys)-\sum_x E^x(sys)
\end{equation}
In the case where relative energies between different conformations of
the same molecule are needed,
$(E^{BSSE}_{at})_{rel}\equiv(E^{BSSE})_{rel}$. 

If the calculation of the full system is performed without spin
polarization, the total energy of each atom will also be calculated
without spin polarization (and vice versa). In this case, specially
when performing the \textit{scf} cycle with HF, symmetry breaking of
the electronic configuration of certain atoms may occur, which might
lead to wrong conclusions. 

The keyword is currently implemented for use with cluster geometries
only. Both RPA and MP2 as the \keyword{total\_energy\_method} are
supported. 

\keydefinition{hydro\_cut}{control.in}
{
  \noindent
  Usage: \keyword{hydro\_cut} \option{flag} \\[1.0ex]
  Purpose: Determines whether or not hydrogenic functions are subject
    to a numerical cutoff potential. \\[1.0ex] 
  \option{flag} is a logical expression, either \texttt{.true.} or
    \texttt{.false.} Default: \texttt{.true.} \\
}
This tag should be kept at the default value unless for testing
purposes (e.g., comparing to other codes). 

\keydefinition{k\_grid}{control.in}
{
  \noindent
  Usage: \keyword{k\_grid} \option{n1} \option{n2} \option{n3} \\[1.0ex]
  Purpose: Sets up an evenly split k-points grid along the reciprocal
    lattice vectors of a periodic calculation \\[1.0ex]
  \option{n1}, \option{n2}, \option{n3} : integer numbers defining the
    number of k-point splits along the first, second and third
    reciprocal axis of the first Brillouin zone, respectively \\
}
\emph{Note} that the order of \option{n1}, \option{n2}, \option{n3}
must \emph{directly} correspond to the order in which the
\keyword{lattice\_vector}s are listed in \texttt{geometry.in}, through
the definition and order of the reciprocal lattice. By
default, the resulting \keyword{k\_grid} is centered around the
$\Gamma$-point, but can be shifted using the \keyword{k\_offset}
keyword below.

The keyword \keyword{symmetry\_reduced\_k\_grid} now allows to make use of
time-reversal symmetry to reduce the number of $k$ points by a factor of
nearly two. This option is the default.

\keydefinition{k\_offset}{control.in}
{
  \noindent
  Usage: \keyword{k\_offset} \option{f1} \option{f2} \option{f3} \\[1.0ex]
  Purpose: Defines a possible non-$\Gamma$ offset for the k-point grid
    in periodic boundary conditions. \\[1.0ex]
  \option{f1}, \option{f2}, \option{f3} : Fractional coordinates
    (between zero and one) of a k-point offset , in units of the
    reciprocal lattice vectors. Default: (0., 0., 0.). \\
}
Can be used to shift the grid off-$\Gamma$ for better $k$-space
sampling. For example, (0.5, 0.5, 0.5) together with even \option{n$i$} for
\keyword{k\_grid} defines a Monkhorst-Pack
\cite{Monkhorst76} type grid.  See Sec.~\ref{sec:k-point-sampling} for
details.

The \keyword{k\_offset} cannot be used with Hartree-Fock or hybrid density functionals
as the current implementation assumes a $\Gamma$-centered grid.

\keydefinition{k\_points\_external}{control.in}
{
  \noindent
  Usage: \keyword{k\_points\_external} \\[1.0ex]
  Purpose: Instead of an internally specified k-point grid, allows to
    specify an externally read $k$-grid from a file
    \texttt{k\_list.in} .\\
}
This option is useful
to specify an uneven special $k$-point set \cite{Chadi73}, etc.

The \keyword{k\_points\_external} keyword cannot presently be used with 
Hartree-Fock or hybrid density functionals
as the current implementation the presence of the
\keyword{k\_grid} keyword and information in \texttt{control.in}.

If specified, \keyword{k\_points\_external} expects a separate input
file \texttt{k\_list.in} to be provided in the same working directory
as all other input files. The format of \texttt{k\_list.in} is as
follows: \\[1.0ex] 
\begin{tabular}{ll}
\emph{line 1:} & \texttt{n1   n2   n3} \\
\emph{line 2:} & \texttt{Nk} \\
\emph{lines 3 - Nk+2}: & \texttt{k1   k2   k3   weight} \\
\end{tabular}

All lines are read as free format. \\
\texttt{n1, n2, n3} are integers,
specifying the descriptors of a 3D $k$-point grid. During s.c.f.,
these values are for reference only. However, they may be used 
for postprocessing after the s.c.f. cycle is complete, e.g., for keyword \keyword{output}
\subkeyword{output}{postscf\_eigenvalues}, where the external k-point
list \texttt{k\_list.in} is not supported. \\
\texttt{Nk} is the (integer) number of $k$-points
following in the file. \\
For each $k$-point, a separate line is
expected, including via \texttt{k1, k2, k3} the coordinates of this
point in units of the reciprocal lattice vectors, and via
\texttt{weight} the integration weight of this $k$-point in all
Brillouin zone integrals.

\keydefinition{onsite\_accuracy\_threshold}{control.in}
{
  \noindent
  Usage: \keyword{onsite\_accuracy\_threshold} \option{threshold} \\[1.0ex]
  Purpose: Issues a warning if the onsite integral of a basis function on 
    the 'radial' integration grid differs from the same onsite integral on
    the more accurate logarithmic grid by more than \option{threshold}. \\[1.0ex]
  \option{threshold} is given in eV. Default: 0.03~eV. \\
}
See keyword \keyword{output} \subkeyword{output}{onsite\_integrands}
for the integrals that are actually evaluated.

For ``light'' settings, you can most likely ignore the associated warning. 
Possibly, treat it as a reminder to check final results with ``tight'' 
settings or similar, just in case.

FHI-aims writes the respective onsite integral values for all its
radial functions side by side after the setup of all radial functions
is complete. Too large deviations between the calculated values on the
radial and logarithmic grids can indicate accuracy problems, which is
a particular concern for high-accuracy benchmark calculations. In
particular, Gaussian-type orbital basis sets of the Dunning type that
are used for high-level reference calculations need grids that are far
more accurate than normal NAO basis sets, due to the unphysical
wiggles that contracted Gaussian functions with high exponents
introduce near the nucleus.

If the associated warning strikes, the grid accuracy can be improved
either using the \subkeyword{species}{radial} keyword or (simpler) the
\subkeyword{species}{radial\_multiplier} keyword.

However, if \keyword{onsite\_accuracy\_threshold}
triggers a warning for ``light'' settings, most likely you can safely
ignore the warning. With light settings, the point may be to do a
reduced-accuracy calculation, which should still be safe for its
original purposes. Just check the radial grid when in doubt.

If the flag \keyword{override\_integration\_accuracy} is toggled to
\texttt{.false.}, however, FHI-aims does stop whenever it encounters a
radial function whose onsite integrals are not deemed accurate enough
by the \keyword{onsite\_accuracy\_threshold} criterion.

\keydefinition{override\_integration\_accuracy}{control.in}
{
  \noindent
  Usage: \keyword{override\_integration\_accuracy} \option{flag} \\[1.0ex]
  Purpose: If set to false, FHI-aims stops calculations for which the
    onsite integral of any radial function is less accurate than
    prescribed by keyword \keyword{onsite\_accuracy\_threshold}. \\[1.0ex]
  \option{flag} can be \texttt{.true.} or \texttt{.false.} Default: \texttt{.true.} \\
}

\keydefinition{symmetry\_reduced\_k\_grid}{control.in}
{
  \noindent
  Usage: \keyword{symmetry\_reduced\_k\_grid} \option{flag} \\[1.0ex]
  Purpose: Determines whether or not to make use of time-reversal symmetry.
  \\[1.0ex] 
  \option{flag} is a logical expression, either \texttt{.true.} or
    \texttt{.false.} Default: \texttt{.true.} \\
}
% JW: There are crucial things needed in order to really make use of symmetry
% for the k-grid which are not implemented in FHI-aims.  FHI-aims can indeed
% recognize symmetries, but it is not able to make use of them.  Should
% someone implement proper symmetries, the following can be uncommented again:

% FHI-aims can now recognize symmetries of the unit cell if they belong to the
% octahedral group, and reduce the number of necessary k-points
%   accordingly. Optionally, the criterion for declaring two atoms
%   symmetry-equivalent can be set by keyword \keyword{symmetry\_thresh} below. 

% \emph{While this is simple functionality and has passed our tests, this option
%   has not seen widespread testing - simply because very often, the number of
%   k-points is not the primary efficiency bottleneck in FHI-aims. We would
%   appreciate any feedback on this option before making it the default
%   behaviour.}

% \emph{VB: Update for version 051610. Symmetry reduction should not be
%   used in production. We have reason to believe that there is an inaccuracy
%   hidden in the implementation. Use non-symmetrized grids instead, the penalty
%   is not large for FHI-aims, unlike in plane wave codes.}


% JW: Not used for any purpose right now:

% \keydefinition{symmetry\_thresh}{control.in}
% {Usage: \keyword{symmetry\_thresh}
%   \option{thresh} \\[1.0em]
% Purpose: To define the maximally allowed coordinate difference (in
% \AA) for the symmetry checker to decide that two atoms are at
% equivalent. Default: 10$^{-6}$\\}
% See keyword \keyword{symmetry\_reduced\_k\_grid} above.

\keydefinition{wave\_threshold}{control.in}
{
  \noindent
  Usage: \keyword{wave\_threshold} \option{threshold} \\[1.0ex]
  Purpose: Determines the outer radius beyond which a radial function
  is considered zero. \\[1.0ex]
  \option{threshold} is a small positive number. Default: 10$^{-6}$.
  Lowered to 10$^{-8}$ by default if \keyword{output}
  \subkeyword{cube} is requested. \\ 
}
A radial function is considered zero (not evaluated) beyond the radius
where $u(r)$ and its first and second derivatives become smaller than
\option{threshold}. The default is chosen such as to not affect any
results at all.

For electron densities or orbitals plotted for visualization using
cube files (\keyword{output} \subkeyword{cube} functionality), a too
high value of \keyword{wave\_threshold} can sometimes lead to small
but visible discontinuities. Thus, the default threshold is
lowered to 10$^-8$ if \keyword{output} \subkeyword{cube} is requested.


\newpage

\subsection*{Subtags for \emph{species} tag in \texttt{control.in}:}

\subkeydefinition{species}{basis\_acc}{control.in}
{
  \noindent
  Usage: \subkeyword{species}{basis\_acc} \option{threshold} \\[1.0ex]
  Purpose: Technical cutoff criterion for on-site orthonormalization
    of radial functions \\[1.0ex]
  \option{threshold} is a small positive real threshold. Default:
  10$^{-4}$. \\
}
Before any calculation, all radial functions for a single species are
Gram-Schmidt orthonormalized. If the norm of the function after
  orthonormalization is smaller than \option{threshold}, that function
  is omitted.

\subkeydefinition{species}{basis\_dep\_cutoff}{control.in}
{
  \noindent
  Usage: \subkeyword{species}{basis\_dep\_cutoff} \option{threshold} \\[1.0ex]
  Purpose: Basis function dependent adjustment of the confinement
  potential for this species \\[1.0ex]
  \option{threshold} is either a positive real number, or can be
  explicitly set \texttt{.false.}. Default: 10$^{-4}$. \\
}
If not \texttt{.false.}, the onset of the basis confining potential
(see \subkeyword{species}{cut\_pot} tag below) is adjusted separately
  for each basis function, such that the norm of this basis function
  \emph{outside} $r_\text{onset}$ is smaller that
  \option{threshold}. The \emph{maximum} possible onset radius is
  still given by the value explicitly specified by the
  \subkeyword{species}{cut\_pot} tag. 

\subkeydefinition{species}{confined}{control.in}
{
  \noindent
  Usage: \subkeyword{species}{confined} \option{n} \option{l}
    \option{radius} \\[1.0ex]
  Purpose: Adds a confined free-atom like radial function to the basis
    set. \\[1.0ex]
  \option{n} is the (integer) radial quantum number. \\
  \option{l} is a character, specifying the angular momentum (
    \emph{s}, \emph{p}, \emph{d}, \emph{f}, ...). \\
  \option{radius} is the onset radius of the confining potential
    (in atomic units, 1~a.u. = 0.529177~\AA). If the word
    \texttt{auto} is specified \emph{instead} of a numerical value,
    the default onset radius given in the
    \subkeyword{species}{cut\_pot} tag is used. \\
}
The defining potential for this basis function type consists of the
non-spinpolarized, self-consistent spherical free-atom potential
(possibly itself confined, using the
\subkeyword{species}{cut\_free\_atom} tag), and a confining
potential. The shape of the confining potential is the same for all
basis functions of a given species, and set using the
\subkeyword{species}{cutoff\_type} and \subkeyword{species}{cut\_pot} 
subtags. 

\subkeydefinition{species}{core}{control.in}
{
  \noindent
  Usage: \subkeyword{species}{core} \option{n} \option{l} \\[1.0ex]
  Purpose: Defines the top ``core'' shell of the species for this
    angular momentum. \\[1.0ex]  
  \option{n} is the (integer) radial quantum number. \\
  \option{l} is a character, specifying the angular momentum (
    \emph{s}, \emph{p}, \emph{d}, \emph{f}, ...). \\
}
Currently not needed for production calculations, but listed here
because the ``core'' infrastructure is currently being reworked and
may see useful additions in the near future. This flag defines 
which electrons of the species are considered ``core'' electrons, and
which enter as explicit valence electrons.

\subkeydefinition{species}{core\_states}{control.in}
{
  \noindent
  Usage: \subkeyword{species}{core\_states} \option{number} \\[1.0ex]
  Purpose: Independent determination of the number of states that are
    core states in the current species. \\[1.0ex]
  \option{number} is an integer number (to be multiplied by 2 for the
    number of core electrons). Default: 0. \\
}
\emph{Experimental at present, not needed for any production
  purposes.} See also the \subkeyword{species}{core} keyword. The
\subkeyword{species}{core\_states} keyword should interact with the
\subkeyword{species}{core} keyword, but does not yet, since any
associated functionality is still under development. Both keywords are
listed here for future reference only.


\subkeydefinition{species}{cut\_atomic\_basis}{control.in}
{
  \noindent
  Usage: \subkeyword{species}{cut\_atomic\_basis} \option{flag} \\[1.0ex]
  Purpose: \textbf{Only} relevant to decide whether the
    \subkeyword{species}{basis\_dep\_cutoff} keyword also applies to
    atomic-type (minimal) radial functions. \\[1.0ex]
  \option{flag} is a logical expression, either \texttt{.true.} or
    \texttt{.false.} Default: \texttt{.false.} \\
}
\textbf{This keyword applies only to the setting specified by keyword}
\subkeyword{species}{basis\_dep\_cutoff}. Do \textbf{not} enable it
routinely without thorough testing.

By default, the minimal basis functions in FHI-aims \emph{are} subject
to the cutoff potential with the fixed onset specified by the
\subkeyword{species}{cut\_pot} keyword. However, the more restrictive
\subkeyword{species}{basis\_dep\_cutoff} keyword does \emph{not} apply
to the minimal basis by default.

This can be changed by setting
\subkeyword{species}{cut\_atomic\_basis} to \texttt{.true.}, but the
associated total energy changes are significantly larger than for
other basis functions. By default, we therefore do not recommend
adding a tighter cutoff to the minimal basis functions at this
time. It is, however, possible that this effect is mainly a systematic
error between the core states, and after further testing, we may yet
choose to enable this keyword \emph{if} we can guaranteed that its
use is safe.

\subkeydefinition{species}{cut\_pot}{control.in}
{
  \noindent
  Usage: \subkeyword{species}{cut\_pot} \option{onset} \option{width}
    \option{scale} \\[1.0ex]
  Purpose: Specifies the numerical parameters for the general
    (default) confinement potential $v_c(r)$ for all basis functions of this
    species. \\[1.0ex]
  \option{onset} specifies the default onset radius of the cutoff
    potential, in {\AA} ($v_c(r)$=0 for $r<r_\text{onset}$). \\
  \option{width} specifies the radial width $w$ of the cutoff potential,
    in {\AA} ($v_c(r)$=$\infty$ for $r>r_\text{onset}+w$). \\
  \option{scale} is a scaling parameter to increase or decrease the
    numerical value of $v_c$. \\
}
This tag is mandatory, since it specifies \option{onset}, a critical
parameter that allows to tune the efficiency of a calculation for
a given target 
accuracy. Unless reduced by the
\subkeyword{species}{basis\_dep\_cutoff} tag, \option{onset} is the
default onset radius used to construct all valence (minimal) and
hydrogen-like basis functions of this species. In addition, any
confined free-atom or free-ion like radial functions use this onset
radius if \texttt{auto} is used in their specification.

Notes: The functional form of $v_c(r)$ can be selected using the
\subkeyword{species}{cutoff\_type} keyword, and \option{width}
and \option{scale} apply to this shape. Modifying these latter
parameters is usually not necessary for a production calculation, but
the \option{onset} value should be verified at least as a quick
numerical check.

\subkeydefinition{species}{cutoff\_type}{control.in}
{
  \noindent
  Usage: \subkeyword{species}{cutoff\_type} \option{identifier} \\[1.0ex]
  Purpose: Specifies the functional form of the confinement potential
    associated with this species. \\[1.0ex]
  \option{identifier} is a string that selects a given confinement
    potential shape as specified in the code. Default: \texttt{exp(1\_x)\_(1-x)2}. \\
}
All confinement potentials in FHI-aims are characterized by the rigorous
boundaries $v_c(r)$=0 for $r<r_\text{onset}$ and $v_c(r)$=$\infty$ for
$r>r_\text{cut} = r_\text{onset}+w$, where $r_\text{onset}$ may depend on the basis
function, and $w$ is the \option{width} specified by the
  \subkeyword{species}{cut\_pot} tag. In addition, each shape contains
  a scaling parameter $s$, also specified via the
  \subkeyword{species}{cut\_pot} tag. 

Available confinement potential shapes (\option{identifier}) for
$r_\text{onset}<r< r_\text{cut} = r_\text{onset}+w$ are:
\begin{itemize}
  \item \texttt{exp(1\_x)\_(1-x)2} : 
    \[
       v_c(r) = \exp(\frac{w}{r-r_\text{onset}}) \cdot \frac{1}{(r-r_\text{cut})^2} 
    \]
    (the default in FHI-aims)
  \item \texttt{junquera} : 
    \[
       v_c(r) = \exp(\frac{w}{r-r_\text{onset}}) \cdot \frac{1}{(r-r_\text{cut})} 
    \]
    (the form originally suggested by Junquera et al. \cite{Junquera01})
  \item \texttt{x2\_(1-x2)}
    \[
       v_c(r) = (r-r_\text{onset})^2 \cdot \frac{1}{(r-r_\text{cut})^2} 
    \]
\end{itemize}


\subkeydefinition{species}{gaussian}{control.in}
{
  \noindent
  Usage: \subkeyword{species}{gaussian} \option{L} \option{N}
    \option{[alpha]} \\
    \hspace*{0.1\textwidth} [ alpha\_1  coeff\_1 ] \\
    \hspace*{0.1\textwidth} [ alpha\_2  coeff\_2 ] \\
    \hspace*{0.1\textwidth} [ ... ] \\
    \hspace*{0.1\textwidth} [ alpha\_N  coeff\_N ]
    \\[1.0ex]
  Purpose: Adds a Gaussian-based radial function to the basis
    set. \\[1.0ex]
  Restriction: This basis function type is not subject to a cutoff
    potential. It may therefore require a wider
    \subkeyword{species}{radial\_base} integration grid than the standard
    NAO's in FHI-aims. \\[1.0ex]
  \option{L} is an integer number, specifying the angular momentum \\ 
  \option{N} is an integer number, specifying how many primitive Gaussians
    comprise the present radial function \\
  \option{alpha} : \emph{If} \option{N}=1, this is the exponent
    defining a primitive Gaussian function [in bohr$^{-2}$]. \\
  \option{alpha\_i} \option{coeff\_i} : 
    \emph{If} \option{N}$>$1, $i=1,\dots,N$ additional lines specify
    exponents $\alpha_i$ and expansion coefficients $g_i$ for a
    non-primitive linear combination of Gaussians.\\
}
FHI-aims allows to use Gaussian-based radial functions to compare to
existing popular Gaussian-based implementations of quantum
chemistry. These functions can either be \emph{primitive} Gaussians, 
\begin{equation}\label{Eq:primGauss}
 u(r) = \frac{1}{\mathit{Norm}} r^{L+1} \cdot \exp(-\alpha r^2) ,
\end{equation}
or \emph{non-primitive} linear combinations.
\begin{equation}\label{Eq:nonprimGauss}
 u(r) = \frac{1}{\mathit{Norm}} r^{L+1} \cdot \sum_{i=1}^{i=N} g_i \exp(-\alpha_i r^2) .
\end{equation}
In existing quantum chemistry codes, Gaussian basis functions can be
defined either as \emph{spherical} Gaussians [Eq. (\ref{Eq:primGauss})
  above], or as \emph{cartesian} Gaussians,
\begin{equation}
 \phi(\boldr) = x^k y^m z^n \exp(-\alpha r^2) , \quad \mathrm{where} \quad
 k+m+n=L .
\end{equation}
This behavior can be mimicked using the
\subkeyword{species}{pure\_gauss} tag (see below). Finally, note that
in order to use an \emph{exclusively} Gaussian-based basis set, you
must prevent the use of the minimal free-atom like NAO basis functions
using the \subkeyword{species}{include\_min\_basis} tag.


\subkeydefinition{species}{hydro}{control.in}{
{
  \noindent
  Usage: \subkeyword{species}{hydro} \option{n} \option{l}
    \option{z\_eff} \\[1.0ex]
  Purpose: Adds a hydrogen-like radial function to the basis
    set. \\[1.0ex]
  \option{n} is the (integer) radial quantum number. \\
  \option{l} is a character, specifying the angular momentum (
    \emph{s}, \emph{p}, \emph{d}, \emph{f}, ...). \\
  \option{z\_eff} scales the radial function as an effective nuclear
    charge in the defining Coulomb potential $z_\text{eff}/r$. \\
}
By default, hydrogen-like basis functions in FHI-aims are subject to
the numerical confinement potential given by
\subkeyword{species}{cutoff\_type} and
\subkeyword{species}{cut\_pot}. Optionally, analytical hydrogen-like
functions (no confinement) can be requested using the global
\keyword{hydro\_cut} tag.
}

\subkeydefinition{species}{include\_min\_basis}{control.in}
{
  \noindent
  Usage: \subkeyword{species}{include\_min\_basis} \option{flag} \\[1.0ex]
  Purpose: Allows to exclude the minimal basis of free-atom valence
    functions from the basis set. \\[1.0ex]
  \option{flag} is a logical expression, either \texttt{.true.} or
    \texttt{.false.} Default: \texttt{.true.} \\
}
This flag is normally only useful to compare explicitly with basis
sets from other methods, usually Gaussian basis sets.

\subkeydefinition{species}{ion\_occ}{control.in}
{
  \noindent
  Usage: \subkeyword{species}{ion\_occ} \option{n} \option{l}
    \option{occupation} \\[1.0ex]
  Purpose: Specifies the shell occupation of a radially symmetric,
    non-spinpolarized free ion that defines any ionic basis
    functions. \\[1.0ex]
  Restriction: Only one type of ion can be used to define ionic basis
    functions for a given species. \\[1.0ex]
  \option{n} is the (integer) radial quantum number. \\
  \option{l} is a character, specifying the angular momentum (
    \emph{s}, \emph{p}, \emph{d}, \emph{f}, ...). \\
  \option{occupation} is the number of electrons in the topmost
    occupied valence shell of this ion. \\
}
This tag defines an ionic configuration, but does not actually add any
ionic functions to the basis used in the present calculation. Actual
basis functions are added by the \subkeyword{species}{ionic}
keyword.

Only the topmost valence shell of each angular momentum channel of the
ion is specified. All lower-lying shells are assumed to be completely
filled for the self-consistent spherical free-ion calculation.

\subkeydefinition{species}{ionic}{control.in}
{
  \noindent
  Usage: \subkeyword{species}{ionic} \option{n} \option{l}
    \option{radius} \\[1.0ex]
  Purpose: Adds a free-ion like radial function to the basis set. \\[1.0ex]
  \option{n} is the (integer) radial quantum number. \\
  \option{l} is a character, specifying the angular momentum (
    \emph{s}, \emph{p}, \emph{d}, \emph{f}, ...). \\
  \option{radius} is the onset radius of the confining potential
    (in atomic units, 1~a.u. = 0.529177~\AA). If the word
    \texttt{auto} is specified \emph{instead} of a numerical value,
    the default onset radius given in the
    \subkeyword{species}{cut\_pot} tag is used. \\
}

\subkeydefinition{species}{pure\_gauss}{control.in}
{
  \noindent
  Usage: \subkeyword{species}{pure\_gauss} \option{flag} \\[1.0ex]
  Purpose: If \texttt{.true.}, any \subkeyword{species}{gaussian}
    basis functions for this species will be purely spherical
    Gaussians. \\[1.0ex]
  \option{flag} is a logical string, either \texttt{.false.} or
    \texttt{.true.} Default: \texttt{.true.} \\
}
See keyword \subkeyword{species}{gaussian} for the distinction between
spherical and cartesian Gaussian functions. In short, cartesian and
spherical Gaussian functions are equivalent \emph{except} that for a
given $L$, cartesian Gaussians add a so-called angular momentum
contamination. If \subkeyword{species}{pure\_gauss} is specified, this
angular momentum contamination is mimicked by FHI-aims.

Consider the simple (textbook!) three-dimensional harmonic oscillator
in quantum mechanics. This can be solved either in cartesian
coordinates, or in spherical coordinates. If solves in cartesian
coordinates, you will find that there are six degenerate solutions for
the principal quantum number 2, five of which correspond to $l$=2 ($d$
channel), but one of which corresponds to $l$=0 ($s$ channel). This
is the exact angular momentum contamination exhibited by the cartesian
definition of a Gaussian basis function. 

\subkeydefinition{species}{valence}{control.in}
{
  \noindent
  Usage: \subkeyword{species}{valence} \option{n} \option{l}
    \option{occupation} \\[1.0ex]
  Purpose: Specifies the shell occupation of the radially symmetric,
    non-spinpolarized free atom that defines the minimal basis. \\[1.0ex]
  \option{n} is the (integer) radial quantum number. \\
  \option{l} is a character, specifying the angular momentum (
    \emph{s}, \emph{p}, \emph{d}, \emph{f}, ...). \\
  \option{occupation} is the number of electrons in the topmost
    occupied valence shell of this ion. \\
}
Only the topmost valence shell of each angular momentum channel of the
atom is specified. All lower-lying shells are assumed to be completely
filled for the self-consistent spherical free-atom calculation. The
valence occupation must be defined explicitly for each \keyword{species}.

The self-consistent free-atom potential generated by this calculation
is used to generate all minimal and \subkeyword{species}{confined}
basis functions used for this species, after the confining potential
is added. 

The self-consistent free-atom calculation can itself be confined by a
different confining potential, the onset of which is specified by the
\subkeyword{species}{cut\_free\_atom} keyword.

For DFT-LDA/GGA, the same \keyword{xc} functional that is used
in the full three-dimensional calculation is also used to define the
self-consistent free atom. For any methods involving Hartree-Fock
exchange (e.g., hybrid functionals), the free atom is generated using
the \option{pw-lda} LDA functional.

The self-consistent free atom density generated here is also used in
the construction of partition functions for the Hamiltonian integrals
and the Hartree potential, as well as to build the initial charge
density (unless otherwise requested!) and the reference charge density
subtracted before constructing the Hartree potential.

\subkeydefinition{species}{sto}{control.in}
{
  \noindent
  Usage: \subkeyword{species}{sto} \option{n} \option{l} \option{zeta}
    \\[1.0ex]
  Purpose: Adds a Slater-type orbital to the basis set. \\[1.0ex]
  Notes: This basis function type is not subject to a cutoff potential. \\[1.0ex]
  \option{n} is an integer which plays the role of the principal quantum number for the STO\\
  \option{l} is an integer specifying the STO angular momentum \\
  \option{zeta} is a double precision number specifying the STO exponent, which plays the role of the effective nuclear charge\\
}

% For emacs:
%%% Local Variables: 
%%% mode: latex
%%% TeX-master: "FHI-aims"
%%% End: 

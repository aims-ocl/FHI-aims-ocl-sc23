\section{A very quick guide to ensuring numerical convergence with FHI-aims}

FHI-aims is programmed and set up to allow efficient all-electron calculations
for any type of system. During the writing of FHI-aims, a key goal was to
always ensure that such efficiency does not come at the price of some
irretrievable accuracy loss. Results obtained by FHI-aims should be
\emph{the} answer to the physical question that you asked (provided that the
functionality is there in FHI-aims) - not some arbitrary approximation. 

The \emph{species\_default} levels provided by FHI-aims, \emph{light},
\emph{intermediate},  
\emph{tight}, and (if ever needed!) \emph{really\_tight}, should provide such
reliable accuracy as they come. The \emph{NAO-VCC-nZ} basis sets
provide additional functionality specifically for correlated methods
(MP2, RPA, $GW$, etc.) and light elements. However, in all
\emph{species\_default} files, all important accuracy choices are
deliberately kept out in the open and available: They can---and
sometimes should!--- be explicitly tested by the user to check the
convergence of a given calculation.  

Such a convergence test may sometimes be geared at simply
ensuring numerical convergence explicitly, but equally, it is possible that
some default settings are too tight for a specific purpose, and can be
relaxed in a controlled way to ensure faster calculations for some large
problem. 

In the following, we explain the most important species default settings
explicitly, and comment on how to choose them. We use the \emph{light}
defaults for Oxygen as an example.

\subsection{Basis set}

The key physical choice to ensure converged results in FHI-aims is the list of
radial functions (and their angular momenta) that are used in FHI-aims. Beyond
the \emph{minimal} basis of free-atom like radial functions, we always
recommend to add at least a single set of further radial functions that are
optimized to describe a chemical bond efficiently. These basis functions can
be found as a list (line by line) at the end of each species defaults
file. For Oxygen / light, the list reads like this:
\begin{verbatim}
#  "First tier" - improvements: -699.05 meV to -159.38 meV
     hydro 2 p 1.8
     hydro 3 d 7.6
     hydro 3 s 6.4
#  "Second tier" - improvements: -49.91 meV to -5.39 meV
#     hydro 4 f 11.6
#     hydro 3 p 6.2
#     hydro 3 d 5.6
#     hydro 5 g 17.6
#     hydro 1 s 0.75
#  "Third tier" - improvements: -2.83 meV to -0.50 meV
#     ionic 2 p auto
#     hydro 4 f 10.8
#     hydro 4 d 4.7
#     hydro 2 s 6.8
 [...]
\end{verbatim}
Obviously, only a single set of radial functions (one for each angular
momentum $s$, $p$, $d$) is active (not commented!) beyond the minimal
basis. Since the minimal basis already contains one additional valence $s$
and $p$ function, this choice is often called ``double numeric plus
polarization'' basis set in the literature (where $d$ is a so-called
polarization function as it does not appear as a valence angular momentum of
the free atom). We call this level ``tier 1''. 

In order to increase the accuracy of the basis, further radial functions may
be added, simply by uncommenting more lines \emph{in order}! We
recommend to normally proceed in order of full ``tiers'', not function by
function, but adding specific individual functions on their own
can sometimes capture the essence of a problem at lower cost.  For
example, tier 2 may be added by uncommenting: 
\begin{verbatim}
#  "First tier" - improvements: -699.05 meV to -159.38 meV
     hydro 2 p 1.8
     hydro 3 d 7.6
     hydro 3 s 6.4
#  "Second tier" - improvements: -49.91 meV to -5.39 meV
     hydro 4 f 11.6
     hydro 3 p 6.2
     hydro 3 d 5.6
     hydro 5 g 17.6
     hydro 1 s 0.75
#  "Third tier" - improvements: -2.83 meV to -0.50 meV
#     ionic 2 p auto
#     hydro 4 f 10.8
#     hydro 4 d 4.7
#     hydro 2 s 6.8
 [...]
\end{verbatim}
tier 2 is the default choice of our \emph{tight} settings for O, but
may be very expensive for hybrid functionals. Look to the
\emph{intermediate} settings for a more economical choice in that case.

Beyond the choice of the radial functions itself, a critical parameter is the
choice of the \emph{confinement} radius that all basis functions
experience. Ensuring that each radial function goes to zero in a controlled
way beyond a certain, given value is critical for numerical efficiency, but on
the other hand, you do not want to reduce this confinement radius too much in
order to preserve the \emph{accuracy} of your basis set.

By default, the confinement radius of each potential is specified by the
following line:
\begin{verbatim}
    cut_pot             3.5  1.5  1.0
\end{verbatim}
This means (see also the CPC publication on FHI-aims, Ref. \cite{Blum08}) that each radial
function is constructed with a confinement potential that begins 3.5~{\AA}
away from the nucleus, and smoothly pushes the radial function to zero over a
width of 1.5~{\AA}. The full extent of each radial function is thus 5~{\AA}. 

Of course, this setting is chosen to give good total energy accuracy at the
\emph{light} level, but the convergence of the confinement potential
must still be tested, especially in situations where a strong
confinement may be unphysical. Such questions include: 
\begin{itemize}
  \item Accurate free atom calculations for reference purposes: choose 8~{\AA}
  or higher for the onset of the confinement, or something similarly
  high---for a single free atom, the CPU time will not matter, and
  you will get
  all the tails of your radial functions right without much thinking.
  \item Surfaces--- e.g., low electron densities above the surface for STM
  simulations must not be abbreviated by the onset of the confinement
  potential---even if the total energy is not affected by this confinement any
  more. 
  \item Neutral alkali atoms, or any negatively charged ions. Those are
  tricky---the outermost electon shell may decay very slowly to zero with
  distance, and explicit convergence tests are required.
\end{itemize}

As the corresponding \emph{tight} setting, we use:
\begin{verbatim}
    cut_pot             4.0  2.0  1.0
\end{verbatim}
Although the modification does not seem large, CPU times for periodic systems
\emph{are} significantly affected by this change of the full extent of each
radial function from 5~{\AA} to 6~{\AA}. For example, in a densely
packed solid, the \emph{density} of basis functions per volume
increases as $R^3$ with the full extent of each radial function, and
thus the time to set up the Hamiltonian matrix should increase as
$R^6$. Very often, the effect on the total
energy is completely negligible, but again, explicit convergence tests are
always possible to make sure.

Finally, there is the line
\begin{verbatim}
    basis_dep_cutoff    1e-4
\end{verbatim}
If this criterion is set above zero (10$^{-4}$ in our \emph{light} settings),
all radial functions are individually checked, and their tails are cut off at
a point where they are already near zero.

You should note that the \texttt{basis\_dep\_cutoff} criterion usually does
not matter at all, but for very large systematic basis set convergence studies
(going to tier 3, tier 4, etc, and/or testing the cutoff potential
explicitly), this value should be set to zero---as is done in the
\emph{really\_tight} settings, for example.

\subsection{Hartree potential}

The Hartree potential in FHI-aims is determined by a multipole decomposition
of the electron density. The critical parameter here is the order (highest
angular momentum) used in the expansion (all higher components are
neglected). This value is chosen by:
\begin{verbatim}
    l_hartree           4
\end{verbatim}
Energy differences with this choice are usually sub-meV converged also for
large systems, but total energy differences, vibrational frequencies at the
cm$^{-1}$ level etc may require more. Our \emph{tight} settings, 
\begin{verbatim}
    l_hartree           6
\end{verbatim}
provide sub-meV/atom converged \emph{total} energies in all our tests, but
you may simply wish to test for yourself ...

\subsection{Integration grid}

FHI-aims integrates its Hamiltonian matrix elements numerically, on a
grid. However, this is an all-electron code: Performing integrations on an
\emph{even-spaced} grid (as is done in many pseudopotential codes) would
provide terrible integration accuracy near the nucleus (sharply peaked, deep
Coulomb potential and strongly oscillating basis functions). 

Instead, we use what is a standard choice also in other codes: Each atom gets
a series of radial spheres (\emph{shells}) around it, and we distribute a certain number
of actual grid points on each shell. Obviously, increasing the number of grid
points (``angular'' points) on each shell will improve the integration
accuracy, but at the price of a linear increase in computational cost.

The fact that the integration spheres will overlap does not matter---we remedy
this fact automatically by choosing appropriate integration weights
(partitioning of unity, see CPC paper).

The number and basic location of radial shells is chosen by
\begin{verbatim}
    radial_base         36 5.0
    radial_multiplier   1
\end{verbatim}
which means that we here choose 36 grid shells, and the outermost shell is
located at 5~{\AA} (this happens to be the outermost radius of each basis
function, as dictated by the confinement potential).

The \subkeyword{species}{radial\_base} tag allows to increase the radial grid density
systematically by adding shells inbetween those specified in the
\texttt{radial\_base} line. For example, we choose 
\begin{verbatim}
    radial_multiplier   2
\end{verbatim}
in our \emph{tight} species defaults (for all practical purposes, this is
converged), which means that we add one shell between the zero and the
(former) first shell, one between the first and second, etc., and finally one
between the (former) outermost shell and infinity ... two times 36 plus one
shells total.

For an
illustration of the effect of the
\subkeyword{species}{radial\_multiplier} keyword on the density of the
radial grid shells, go to Ref. \cite{Zhang2013}
(\url{http://iopscience.iop.org/1367-2630/15/12/123033/article}, open
access) and look at Figure A.1 and the accompanying explanation. 

The distribution of actual grid points \emph{on} these shells is done using
so-called Lebedev grids, which are designed to integrate all angular
momenta up to a certain order $l$ exactly. They come with fixed numbers of grid
points (50, 110, 194, etc). As a rule, fewer grid points will be needed in the
inner grid shells, and more will be needed at the (more extended) faraway grid
shells. We specify the increase the number of grid points per radial shell in
steps, by writing:
\begin{verbatim}
     angular_grids specified
      division   0.2659   50
      division   0.4451  110
      division   0.6052  194
#      division   0.7543  302
#      division   0.8014  434
#      division   0.8507  590
#      division   0.8762  770
#      division   0.9023  974
#      division   1.2339 1202
#      outer_grid 974
      outer_grid 194
\end{verbatim}
This example pertains to the \emph{light\_194} settings (and very light
elements), and means that only 
50 points will be used on all grid shells inside a radius of 0.2659~{\AA}, 110
grid points are used on all shells within 0.4451~{\AA}, 194 grid points will
be used on all shells inside 0.6052~{\AA}---and that's it! No more
\texttt{division} tags are uncommented, and all shells outside 0.6052~{\AA}
also get 194 grid shells, as given by the uncommented \texttt{outer\_grid}
tag. 

We note that the form of the \emph{increase} of the number of points per
radial shell near the nucleus, as well as the \emph{maximum} number of angular
grid points used outside a given radius are \emph{critical} for the numerical
accuracy in FHI-aims. When suspecting numerical noise anywhere in the
calculations, the specification of the angular grid points should be checked
first. This can be done by uncommenting further \texttt{division} tags with
larger numbers of grid points, as well as a suitably increased
\texttt{outer\_grid} value. In particular, the choice of only 194 grid points
max. per radial shell (only for the lightest elements!) is a rather aggressive
choice, but in our experience still enables very reasonable geometry
relaxations, structure searches or molecular dynamics for most
purposes. However, the first thing to check in order to provide better
convergence would be to set \texttt{outer\_grid} to 302 (regular
\emph{light} settings). If this produces a
noticeable change of the quantity you are calculating, be careful.

Of course, one can always introduce the denser grids provided in the
\emph{tight} settings, which (for reference) are
\begin{verbatim}
     angular_grids specified
      division   0.1817   50
      division   0.3417  110
      division   0.4949  194
      division   0.6251  302
      division   0.8014  434
#      division   0.8507  590
#      division   0.8762  770
#      division   0.9023  974
#      division   1.2339 1202
#      outer_grid 974
      outer_grid  434
\end{verbatim}
These grids alone are roughly twice as expensive as the \emph{light\_194} ones
above, and should provide reasonable accuracy for pretty much any
purpose. Nonetheless, of course one can still go and check explicitly, simply
by increasing the number of grid points per shell by hand. 




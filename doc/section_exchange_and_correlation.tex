\section{Electronic structure: Exchange, correlation (incl. DFT+U), and excited states}
\label{Sec:xc}

A key choice \emph{required} in every electronic structure calculation is the
treatment of the required electronic structure: Exchange, correlation,
and potentially quasiparticle energies, e.g., after a $GW$ correction.

We here summarize the general options available regarding the choice
of the electronic structure method. In addition, an important question
is \emph{which} electrons in the structure are treated at which
level. For most practical purposes, FHI-aims treats all electrons in
an equivalent way, but for some special cases, frozen-core treatments
may be useful: at present, one may compute the correlation
energy of only the \emph{valence} but not the \emph{core} electrons in
second-order M{\o}ller-Plessett (MP2) perturbation theory.

For any method requiring the
two-electron Coulomb operator explicitly (these include hybrid
functionals, Hartree-Fock, MP2 or RPA perturbation theory, $GW$
corrections, etc.) we note that an auxiliary basis is required to
expand the Coulomb matrix (four basis functions $\equiv$ $O(N^4)$
matrix elements) into a two-center Coulomb matrix, leading instead to
$O(N^3)$ additional overlap matrix elements. The choice of this
auxiliary basis (``product basis'') is described in more detail in
Sec. \ref{Sec:auxil} and Ref.~\cite{Ren12a}.

\textbf{A note on ``post-s.c.f'' RPA-based methods }

The algorithms for post-DFT methodologies as implemented in FHI-aims
are detailed in Ref.~\cite{Ren12a}. Here we only briefly 
recapitulate the key ingredients behind the increasingly popular 
``RPA and beyond'' methods as implemented in FHI-aims. The standard RPA
total energy is computed as follows:
 \begin{equation}
   E^\text{RPA}_\text{tot} = E^\text{DFT}_\text{tot} - E^\text{DFT}_\text{xc}
        + E^\text{EX}_\text{x} + E^\text{RPA}_\text{c} \, .
  \label{eq:RPA_energy}
 \end{equation}
$E^\text{DFT}_\text{tot}$ is a pre-computed self-consistent DFT total energy
obtained from LDA, GGA, or hybrid functional calculations. $E^\text{DFT}_\text{xc}$
is the corresponding exchange-correlation contribution. $E^\text{EX}_\text{x}$ and
$E^\text{RPA}_\text{c}$ are the exact-exchange energy, and the RPA non-local 
correlation evaluated using the pre-determined Kohn-Sham or generalized Kohn-Sham
eigenorbitals and eigenenergies.

Recently, several correction schemes to RPA have been proposed. 
FHI-aims currently provides the (renormalized) single excitation (SE) correction
\cite{Ren11} and the second-order screened exchange (SOSEX) correction 
\cite{Grueneis09}. The renormalized SE (rSE) and SOSEX corrections can be combined. The 
combined scheme is called ``renormalized 2nd-order perturbation theory''
(rPT2) \cite{Ren12b},
 \begin{equation}
    E_\text{tot}^\text{rPT2} = E^\text{RPA}_\text{tot} + E^\text{SOSEX}_\text{c} +
         E^\text{rSE}_\text{c}\, .
  \label{eq:rPT2_energy}
 \end{equation}
The ``RPA+SE'', ``RPA+rSE'', and ``RPA+SOSEX'' total energies 
can be computed similarly by combining the corresponding terms.

\textbf{A note on ``DFT plus U''}

In the DFT method with local or semi-local approximations of the
XC-functional, (LDA, GGA, etc.) strongly correlated systems like transition
metal oxides are poorly described. The ``DFT plus U'' method offers an ad hoc
correction for strongly correlated systems at negligible computation 
cost~\cite{Anisimov2000}.

\emph{The present implementation of ``DFT plus U'' in FHI-aims should be
considered experimental, and is not complete in some respects. Please keep
this in mind when using the method. That said, it should give physically
sensible results. Simply take some care when using it, and please give us
feedback if the method works for you (obviously, also if it does not for some
reason).} 
\begin{itemize}
  \item DFT+U total energies can be obtained in combination with any functional
    (typically LDA or GGA), by simply adding appropriate \subkeyword{species}
    {plus\_u} tags to the corresponding species. 
  \item Total energy gradients (``forces'') are not provided. 
  \item The implementation does not yet offer self-consistent determination of
    the U parameter, so this needs to be supplied by hand.
  \item Finally, the orbitals on which we project are the somewhat extended
    free-atom like orbitals, defined with the usual cutoff potential of the
    remaining calculation. While somewhat arbitrary, it would be useful to be
    able to project onto more localized orbitals, but this option is not
    implemented yet.
\end{itemize}

\subsection*{Tags for general section of \texttt{control.in}:}

\keydefinition{frozen\_core}{control.in}
{
  \noindent
  Usage: \keyword{frozen\_core} \option{first\_orbital} \\[1.0ex]
  Purpose: Allows to compute the MP2 correlation energy without the
    contribution arising from low-lying occupied orbitals. \\[1.0ex]
  \option{first\_orbital} is the integer number of the first molecular
    orbital that is \emph{included} in the computation of the MP2
    correlation energy. \\
}
\textbf{This keyword applies only to the calculation of the MP2
  correlation energy (if requested). It does not imply a frozen-core
  treatment anywhere else.}

In a nutshell, this is a simple way to exclude the large contribution
from certain core electrons to the MP2 correlation energy. This
contribution is mostly systematic, and therefore tends to cancel in
energy differences. However, it is also the hardest to compute unless
specialized basis sets are invoked that ``know'' about core
correlation; it may thus be the source of a large systematic error
that also cancels if excluded from the beginning. For consistency
between different calculations, the number of excluded ``core''
orbitals must be readjusted between calculations with different
numbers of atoms.

\keydefinition{frozen\_core\_postscf}{control.in}
{
  \noindent
  Usage: \keyword{frozen\_core\_postscf} \option{valence\_shell\_number} \\[1.0ex]
  Purpose: Alternative way to specify the valence shells in the frozen-core algorithm for MP2, RPA and
  rPT2 methods. \textit{This keyword is valid for elements from H (1) to Rn (86)}.\\[1.0ex]
  \option{valence\_shell\_number} is the number of valence shells which are taken into account  
  in the frozen-core MP2, RPA or rPT2 calculations.\\
}

Compared to the keyword \keyword{frozen\_core}, \keyword{frozen\_core\_postscf} is more friendly, especially for
large systems, as you don't need to count by hand which is the first valence orbital in the frozen-core algorithm.

For example, \emph{valence\_shell\_number=2} means that at most two outer shells are taken as
valence shells in the frozen-core calculations:

\begin{center}
    \begin{tabular}{ccc}
    	\hline
    	element & core shells & valence shells \\
    	\hline
    	H, He   & --          & 1$s$\\
    	Li-Ne   & --          & 1$s$2$s$2$p$\\
    	Na-Ar   & 1$s$        & 2$s$2$p$3$s$3$p$\\
    	K-Kr    & 1$s$2$s$2$p$& 3$s$3$p$3$d$4$s$4$p$\\
    	\hline
    \end{tabular}
\end{center}


\keydefinition{hybrid\_xc\_coeff}{control.in}
{
 \noindent
 Usage: \keyword{hybrid\_xc\_coeff} \option{value} \\[1.0ex]
 Purpose: If set, will modify the (Hartree-Fock) exact exchange mixing parameter in a given 
          hybrid XC functional. \emph{No effect} if specified with a
          simple LDA / GGA type functional.\\[1.0ex]
 \option{value} is a real number (usually between zero and one) that
          specifies the degree of exact exchange admixture.\\
}
\emph{If} (and only if) a hybrid functional is specified using the \keyword{xc} keyword, 
\keyword{hybrid\_xc\_coeff} allows to change the Hartree-Fock mixing parameter to 
a different, given \option{value}. For example, the mixing parameter in \texttt{pbe0}
could be specified away from its literature value, $\alpha$=0.25. No effect for 
\keyword{xc} functionals that do not have any Hartree-Fock exchange admixed in the first place.

Obviously, this option
is only useful for test purposes and does change the definition of any functional
away from its literature value. Handle with care.

\keydefinition{hse\_unit}{control.in}
{
 \noindent
 Usage: \keyword{hse\_unit} \option{character} \\[1.0ex]
 Purpose: Required clarification of units for the \texttt{hse06} \keyword{xc} functional.\\[1.0ex]
 \option{value} is a character, either 'a' or 'A' (for {\AA}$^{-1}$) or 'b' or 'B' (for [bohr radius]$^{-1}$).\\
}
The \texttt{hse06} functional comes with a screening parameter $\omega$ which must be specified explicitly
(see the \keyword{xc} keyword for a detailed explanation). Unfortunately, different codes and authors
appear to have adopted different conventions for $\omega$ -- either {\AA}$^{-1}$ or [bohr radius]$^{-1}$. 
To avoid any possible confusion when using HSE06 in FHI-aims, we therefore only run \texttt{hse06} if the
unit has been explicitly specified, using the above keyword. We apologize for the inconvenience, but
the risk of an innocent misunderstanding is rather high in the present case.

\keydefinition{lc\_dielectric\_constant}{control.in}
{
 \noindent
 Usage: \keyword{lc\_dielectric\_constant} \option{value} \\[1.0ex]
 Purpose: If set, will modify the amount of exact exchange in the hybrid XC functional LC-$\omega$PBEh. 
          
 \option{value} is a real number (larger or equal to one) that
          specifies the degree of exact exchange admixture in the long-range part. default=1.0\\
}

\keydefinition{plus\_u\_petukhov\_mixing}{control.in}
{
  \noindent
  Usage: \keyword{plus\_u\_petukhov\_mixing} \option{mixing\_factor} \\[1.0ex]
  Purpose: \emph{Experimental---only for DFT+U.} Allows to fix the mixing
    factor between AMF and FLL contribution of the double counting
    correction~\cite{Petukhov03}. \\[1.0ex] 
  \option{mixing\_factor} is a floating point value, specifying the mixing
    ratio between 0.0 and 1.0. A value of 0.0 selects the Around Mean Field
    (AMF) contribution. A value of 1.0 selects the Fully Localized Limit
    (FLL). If unspecified, the value is determined self-consistently according
    to Ref.~\cite{Petukhov03}. \\ 
}

There are two common schemes for dealing with the double counting problem in DFT+U: The AMF method
assumes that the effect of the DFT+U term on the actual occupations remains small, so that the
occupations can be assumed to be equal within each shell for the purpose of the double counting
correction. The FLL method, on the other hand, assums a maximal effect of the DFT+U term on the
occupation numbers, handling double counting correctly in the case that all orbitals with in the
shell are either fully occupied or empty. The self consistent mixing of both limits improves the
handling of the intermediate range (see Ref.~\cite{Petukhov03}).

\keydefinition{qpe\_calc}{control.in}
{
 \noindent
 Usage: \keyword{qpe\_calc} \option{selfenergy-type} \\[1.0ex]
 Purpose: If set, specifies which self-energy should be used for a
 quasiparticle correction of single-particle eigenvalues. \\[1.0ex]
 \option{selfenergy-type} is a keyword (string) which specifies the
   selfenergy approximation used. \\
}
\emph{Note} that quasiparticle corrections ($GW$, MP2) are currently possible
only for cluster geometries (no periodic boundary conditions).

After the normal self-consistency cycle for a given
exchange-correlation functional (set using the \keyword{xc} keyword)
is complete, \keyword{qpe\_calc} can be used to specify a perturbative
quasiparticle correction to be applied as a post-processing
step. Valid self-energy options \option{selfenergy-type} are:
\begin{itemize}
  \item \option{gw} : Perturbative $G_0 W_0$-type self-energy, where both the
    Green's function $G_0$ and the screened Coulomb interaction $W_0$ are
    computed only once, based on the self-consistent DFT or Hartree-Fock
    ground state eigenvalues and eigenfunctions.
  \item \option{ev\_scgw}  Perturbative $G_0 W_0$-type self-energy, where 
    self-energy is evaluated with partial self-consistency in the eigenvalues. 
    Molecular orbitals are kept unchanged from the preliminary
    calculation. For true self-consistent $GW$, see the
    \keyword{sc\_self\_energy} 
    further below.
  \item \option{mp2} : Perturbative MP2-type self-energy, based on the
    self-consistent DFT or Hartree-Fock ground state eigenvalues and
    eigenfunctions.
\end{itemize}

\keydefinition{sc\_self\_energy}{control.in}
{
 \noindent
 Usage: \keyword{sc\_self\_energy} \option{self-consistent-scheme} \\[1.0ex]
 Purpose: If set, specifies the scheme adopted for the 
 self-consistent calculation of the many-body self-energy. \\[1.0ex]
 \option{selfenergy-type} is a keyword (string) which specified the 
 self-consistent approach used in the calculation. \\
}
\emph{Note} that self-consistent $GW$ calculation (sc-$GW$, sc-$GW_0$) are currently possible
only for cluster geometries (no periodic boundary conditions).

After the normal self-consistency cycle for a given
exchange-correlation functional (set using the \keyword{xc} keyword)
is complete, \keyword{sc\_self\_energy} can be used to specify
a self-consistent scheme for the calculation of the $GW$ self-energy.
The output consists of the total energy calculated from the Galitskii-Migdal 
formula, an output file (\texttt{spectrum\_sc.dat} for spin unpolarized, \texttt{spectrum\_sc\_up.dat} and 
 \texttt{spectrum\_sc\_do.dat} for spin up and down respectively in the case of spin polarized calculation) 
containing the spectral function calculated from the self-consistent Green's function.
At the end of the calculation, the output include the dipole 
moment evaluated from the self-consistent density.

Currently implemented self-consistent methods are:
\begin{itemize}
  \item \option{scgw} : Calculate the Green's function by solving until full self-consistency the 
  Dyson's equation by using a self-energy in the $GW$ approximation. 
  \item \option{scgw0} : Solve self-consistently the Dyson's equation with the self-energy in the 
   $GW_0$ approximation. Differently from fully self-consistent $GW$, in this case the screened Coulomb 
   interaction is kept fixed at the RPA level.
\end{itemize}

\keydefinition{scgw\_mix\_param}{control.in}
{
 \noindent
  Usage: \keyword{scgw\_mix\_param} \option{$\alpha$} \\[1.0ex]
  Purpose: Define the linear mixing coefficient $\alpha$, for the mixing of the Green function 
  at each iteration of the self-consistent $GW$ calculation.  This keyword only produces an effect if  
  \keyword{sc\_self\_energy} is set.
}

\keydefinition{scgw\_it\_limit}{control.in}
{
 \noindent
  Usage: \keyword{scgw\_it\_limit} \option{$N$} \\[1.0ex]
  Purpose: Set the maximum number $N$ of iteration of the Dyson equation
  in a self-consistent $GW$ calculation. The default value is set to $N=30$. 
  This keyword only produces an effect if
  \keyword{sc\_self\_energy} is set.
}


\keydefinition{scgw\_print\_all\_spectrum}{control.in}
{
 \noindent
  Usage: \keyword{scgw\_print\_all\_spectrum} \\[1.0ex]
  Purpose: 
  Enables the print out of the spectral function each iteration of the self-consistent $GW$ calculation. 
  The spectrum is printed to the file \texttt{sp\_ImG<N>.dat}, where <N> is number of iteration of the Dyson equation. 
  This keyword only produces an effect if \keyword{sc\_self\_energy} is set.
}

\keydefinition{rpa\_along\_ac\_path}{control.in}
{
 \noindent
  Usage: \keyword{rpa\_along\_ac\_path} \option{rpa\_along\_ac\_path\_grid} \\[1.0ex]
  Purpose: Calculate the RPA-approximated potentials along the adiabatic-connection path.\\[1.0ex]
  \option{rpa\_along\_ac\_path\_grid} is the number of potentials you want to sampling
  along the adiabatic-connection path.\\
}
The standard RPA method is a adiabatic-connection advanced DFT method, which
integrates the contribution along the adiabatic-connection path analytically.
This keyword \keyword{rpa\_along\_ac\_path} allows you to unpack the adiabatic-connection
path in the RPA approximation in detail.

\keydefinition{printout\_dft\_components}{control.in}
{
 \noindent
  Usage: \keyword{printout\_dft\_component} \option{given\_dft\_method} \\[1.0ex]
  Purpose: Evaluate the XC contributions of a given DFT method based on SCF converged KS (or HF) orbitals.\\[1.0ex]
  \option{given\_dft\_method} is the name of the DFT method you want to investigate. At present, only two
  GGA methods (PBE and BLYP) are avaiable.\\
}
This keyword \keyword{printout\_dft\_components} is repeatable in the contril.in allowing to inspect several DFT methods
in one task.

\keydefinition{scs\_mp2\_parameters}{control.in}
{
 \noindent
  Usage: \keyword{scs\_mp2\_parameters} \option{pT} \option{pS} \\[1.0ex]
  Purpose: For MP2 correlation energies, allows to perform
    spin-component scaled MP2. \\[1.0ex]
  \option{pT} is the scaling parameter for the spin-up-spin-up
    (triplet) contribution. \\
  \option{pS} is the scaling parameter for the spin-up-spin-down
    (singlet) contribution. \\
}
The MP2 correlation energy (\keyword{total\_energy\_method}
\texttt{mp2} or \keyword{xc} \texttt{mp2}) can
be separated into a sum of triplet (spin-up-spin-up) and singlet
(spin-up-spin-down) two-electron terms:
\begin{equation}
  E_\text{corr,MP2} = E_T + E_S \, .
\end{equation}
Grimme \cite{Grimme03} pointed out that empirical scaling factors
$p_T$ and $p_S$ can be introduced and fitted to improve the accuracy
of MP2 results compared to quantum-chemical benchmark methods:
\begin{equation}
  E_\text{SCS,MP2} = p_T E_T + p_S E_S \, .
\end{equation}
For example, $p_T$=1/3 and $p_S$=6/5 are employed to obtain the
reaction energies of Table I in Ref. \cite{Grimme03}.

\keydefinition{total\_energy\_method}{control.in}
{ \noindent
 Usage: \keyword{total\_energy\_method} \option{type} \\[1.0ex]
 Purpose: If set, specifies an exchange-correlation method \emph{for
   post-processing only}, after the scf cycle is complete. \\[1.0ex]
 \option{type} is a keyword (string) which specifies the chosen
   post-processing exchange-correlation method. \\
}
After the regular scf cycle is complete for a given exchange-correlation
method as given by the \keyword{xc} tag, the resulting Kohn-Sham orbitals and
eigenvalues are used to recalculate \emph{only} the exchange-correlation
energy, and only once (i.e., perturbative post-processing). Valid
post-processing options \option{type} are:
\begin{itemize}
  \item \option{C6\_coef} : Molecular C$_6$ dispersion coefficients at the
    MP2 / RPA level will be calculated after the
    s.c.f. calculation. (This functionality is somewhat experimental,
    be sure to check for consistency.)
  \item \option{hf} or \option{HF}: Calculate Hartree-Fock exchange on the
    given orbitals.
  \item \option{ll\_vdwdf} : The nonlocal part of correlation energy is calculated
    using the van der Waals density functional proposed by M.
    Dion {\it et al.}~\cite{Dion04} and the total correlation
    energy will be re-evaluated as proposed in their paper. For
    details about additional Tags needed for the calculation,
    please visit Sec. \ref{Sec:vdwdf}. Note that an alternative
    implementation by the Helsinki group is available as well, the
    present keyword is not your only option.
  \item \option{m06-l} or \option{M06-L}: Truhlar's optimized local meta-GGA of the ``M06'' suite of
  functionals. \cite{ZhaoTruhlar06_M06-L}
  \item \option{m06} or \option{M06}: Truhlar's optimized hybrid meta-GGA of the ``M06'' suite of
    functionals; with 27\% exact exchange. \cite{ZhaoTruhlar06_M06_M06-2X}
    Currently supported only for non-hybrid functionals specified through the \keyword{xc} tag.
  \item \option{m06-2x} or \option{M06-2X}: Truhlar's optimized hybrid meta-GGA of the ``M06'' suite of
    functionals, with double contribution (54\%) from the hartree-fock exact exchange. \cite{ZhaoTruhlar06_M06_M06-2X}
    Currently supported only for non-hybrid functionals specified through the \keyword{xc}
    tag. 
  \item \option{m06-hf} or \option{M06-HF}: Truhlar's optimized hybrid meta-GGA of the ``M06'' suite of
    functionals, with 100\% exact exchage contribution. \cite{ZhaoTruhlar06_M06-HF}
    Currently supported only for non-hybrid functionals specified through the \keyword{xc} tag.
  \item \option{m08-hx} or \option{M08-HX}: Truhlar's optimized hybrid meta-GGA of the ``M08'' suite of
    functionals, with 52.23\% contribution from the hartree-fock exact exchange. \cite{ZhaoTruhlar08_M08-HX_M08-SO}
    Currently supported only for non-hybrid functionals specified through the \keyword{xc} tag.
  \item \option{m08-so} or \option{M08-SO}: Truhlar's optimized hybrid meta-GGA of the ``M08'' suite of
    functionals, with 56.79\% contribution from the hartree-fock exact exchange. \cite{ZhaoTruhlar08_M08-HX_M08-SO}
    Currently supported only for non-hybrid functionals specified through the \keyword{xc} tag.
  \item \option{m11-l} or \option{M11-L}: Truhlar's optimized range-separated local meta-GGA of the ``M11'' suite of
  functionals. \cite{PeveratiTruhlar11_M11-L}
  \item \option{mp2} : The correlation energy is calculated in second-order
    M{\o}ller-Plesset perturbation theory (MP2), with Hartree-Fock added for
    the exchange part. \emph{Note added in March 2016: A periodic
      implementation of MP2 is available but, at the time of writing,
      computationally extremely expensive. If you decide to use it,
      please keep in mind that the periodic version is included here
      as a matter of protocol but is not yet optimized to be fully
      usable in production calculations.}
  \item \option{pbe\_vdw} : Evaluates the van der Waals density
    functional proposed by M. Dion {\it et al.}~\cite{Dion04} with the
    methodology of Sec. \ref{vdwdf-TKK}. (Uses PBE exchange.) \emph{This is
    \emph{not} the Tkatchenko-Scheffler correction~\cite{TS-vdw}. If
    you are looking for Tkatchenko-Scheffler, please use the keyword
    \keyword{vdw\_correction\_hirshfeld} instead.}
  \item \option{revpbe\_vdw} : As pbe\_vdw but uses revpbe instead of pbe
    for the exchange. \emph{This is also
    \emph{not} the Tkatchenko-Scheffler correction~\cite{TS-vdw}. If
    you are looking for Tkatchenko-Scheffler, please use the keyword
    \keyword{vdw\_correction\_hirshfeld} instead.}
  \item \option{revtpss} : Meta-GGA revTPSS functional, thanks to E. Fabiano
    and F. Della Sala. \cite{Perdew09_revTPSS, Perdew11_revTPSS}
  \item \option{nlcorr} : Only the non-local correlation term of the
    pbe\_vdw or revpbe\_vdw is calculated and added to the total
    energy. \emph{And this is still 
    \emph{not} the Tkatchenko-Scheffler correction~\cite{TS-vdw}. If
    you are looking for Tkatchenko-Scheffler, please use the keyword
    \keyword{vdw\_correction\_hirshfeld} instead.}
  \item \option{rpa} : The RPA total energy as defined in Eq.~(\ref{eq:RPA_energy}) will be calculated. 
    When this option is specified, the SE and rSE corrections to RPA are also
    evaluated. The total enegies computed with the RPA, RPA+SE, and RPA+rSE schemes 
    are listed in items \texttt{``RPA total energy''}, \texttt{``PRA+SE total energy''},
    and \texttt{``RPA+rSE (full) total energy''} respectively in the output file.
  \item \option{rpa+2ox} : Just RPA plus second-order exchange (not
    screened). Likely only useful for testing / benchmarking, use rpt2
    for completeness.
  \item \option{rpa+sosex} : Just RPA plus second-order screened
    exchange. Likely only useful for testing / benchmarking, use rpt2
    for completeness.
  \item \option{rpt2} : The rPT2 total energy as defined in
    Eq.~(\ref{eq:rPT2_energy}) will be calculated. When this option is
    specified, the ``RPA+SOSEX'' total energy without the rSE
    correction will also be printed out in the output file.
  \item \option{scan} or \option{SCAN}: ``Strongly Constrained and
    Appropriately Normed Semilocal Density Functional,'' i.e., the
    SCAN meta-GGA functional by Sun, Ruzsinszky, and Perdew.\cite{Sun2015} \\
    \emph{Note: This functionality is very new at the time of the
      FHI-aims 2016 release in March, 2016. It has been tested, but
      please treat it with appropriate caution for a while.}
  \item \option{tpss} : Meta-GGA TPSS functional, thanks to E. Fabiano
    and F. Della Sala. \cite{Tao03_TPSS}
  \item \option{tpssloc} : Meta-GGA TPSSloc functional, thanks to E. Fabiano
    and F. Della Sala. Reference: L.A. Constantin, E. Fabiano, F.Della Sala,
    Phys. Rev. B. 86, 035130 (2012).
  \item \option{xyg3} : ``XYG3'' double-hybrid functional\cite{Zhang:2009}, which is
    defined only for a self-consistent B3LYP reference, i.e.,
    \keyword{xc} \option{b3lyp} is mandatory. \emph{Note} that double-hybrid
    functionals include MP2 components. When using the \emph{tier} basis sets,
    you must use a counterpoise correction of energy differences to get 
    reliable results.
  \item \option{xdh-pbe0} : ``xDH-PBE0'' double-hybrid functional\cite{Zhang2012}, which is
    defined only for a self-consistent PBE0 reference, i.e.,
    \keyword{xc} \option{pbe0} is mandatory. \emph{Note} that double-hybrid
    functionals include MP2 components. When using the \emph{tier} basis sets,
    you must use a counterpoise correction of energy differences to get 
    reliable results.
  \item \option{dfauto <name>} : \emph{See the \option{dfauto} section for tag \option{xc}.}
\end{itemize}
\emph{Note} that some of the correlation methods available here are only
supported for cluster geometries at this time. 
\emph{Note} also that when advanced correlation methods (e.g. rpa, rpt2, xyg3, xdh-pbe0 and mp2) are 
used for binding energy calculations, a \textbf{counterpoise correction} should always be
performed with the default NAO basis sets in FHI-aims to get reliable results, 
since the basis set superposition error (BSSE) for these correlation methods is significant.
For these advanced correlation methods, the sequence of NAO valence-correlation consistent 
basis sets (\emph{NAO-VCC-nZ}\cite{Zhang2013}) is a better choice, which reduces the basis set incompleteness error,
including BSSE, with increasing the basis size, and especially enables to approach the completeness-basis-set
limit with the aid of extrapolation scheme.

\keydefinition{use\_2d\_corr}{control.in}
{
 \noindent
 Usage: \keyword{use\_2d\_corr} \option{bool} \\[1.0ex]
 Purpose: Specifies whether to use the efficient 2D distribution of the MO
 based three index arrays where possible.  Otherwise, stick to the old 1D
 distribution in all cases.
 \\[1.0ex]
 Default: \option{.true.} \\[1.0ex]
}

\keydefinition{xc}{control.in}
{
 \noindent
 Usage: \keyword{xc} \option{xc-type} [\option{value}] \\[1.0ex]
 Purpose: Specifies the exchange-correlation approach used for self-consistent
 DFT / Hartree-Fock. See also \keyword{xc\_pre}. \\[1.0ex]
 Default: \option{pw-lda} \\[1.0ex]
 \option{xc-type} is a keyword (string) which specifies the chosen
    exchange-correlation functional. \\
 \option{value} is a real parameter needed only for some functionals
 (e.g., \texttt{hse06}). \
}
FHI-aims provides a wide range of current exchange-correlation options,
ranging from local-density and generalized-gradient approximations (LDAs and
GGAs) via hybrid functionals and Hartree-Fock to two-electron
treatments of the correlated many-body system, such as second-order
M{\o}ller-Plesset (MP2) theory and the random-phase approximation (RPA). The
following choices for the \option{xc-type} option are currently available:
\begin{itemize}
  \item Local-density approximation (different parameterizations):
    \begin{itemize}
      \item \option{pw-lda} : Homogeneous electron gas based on Ceperley and
        Alder \cite{Cep80} as parameterized by Perdew and Wang 1992
        \cite{Per92}. \emph{Recommended LDA parameterization.}
      \item \option{pz-lda} : Homogeneous electron gas based on Ceperley and
         Alder \cite{Cep80}, as parameterized by Perdew and Zunger 1981
         \cite{Per81}.
      \item \option{vwn} : LDA of Vosko, Wilk, and Nusair 1980 \cite{Vosko80}.
      \item \option{vwn-gauss} : LDA of Vosko, Wilk, and Nusair 1980, \emph{but
        based on the random phase approximation} \cite{Scuseria05}. Do
        not use this LDA unless for one specific reason: In the B3LYP
        implementation of the Gaussian code, this functional is allegedly
        used instead of the correct VWN functional. It is therefore now
        present in many reference results in the literature, and also
        available here for comparison.
    \end{itemize}
  \item Generalized-gradient approximations:
    \begin{itemize}
      \item \option{am05} : GGA functional designed to include surface effects
    in self-consistent density functional theory, according to Armiento
    and Mattsson \cite{AM05}
      \item \option{blyp} : The BLYP functional: Becke (1988) exchange
      \cite{Becke88b} and Lee-Yang-Parr correlation \cite{Lee88}.
      \item \option{pbe} : GGA of Perdew, Burke and Ernzerhof 1997 \cite{Per97}.
      \item \option{pbeint} : PBEint functional of Ref. \cite{Fabiano10}
      \item \option{pbesol} : Modified PBE GGA according to Ref. \cite{Perdew08}.
      \item \option{rpbe} : The RPBE modified PBE functional according to
        Ref. \cite{Hammer99}.
      \item \option{revpbe} : The revPBE modified PBE GGA suggested in Ref. \cite{Zhang98}.
      \item \option{r48pbe} : The mixed functional containing 0.52*pbe and 0.48*rpbe according to Ref. \cite{Nattino2012}
      \item \option{pw91\_gga} : GGA according to Perdew and Wang,
        usually referred to as  
        ``Perdew-Wang 1991 GGA''. This GGA is most accessibly
        described in References 26 and 27 of Ref. \cite{Perdew92}. Note
        that the often mis-quoted reference \cite{Per92} does
        \emph{not}(!) describe the Perdew-Wang GGA but instead only the
        correlation part of the local-density approximation described above.
    \end{itemize}
  \item Meta-generalized gradient approximations:
    \begin{itemize}
      \item \option{m06-l} : Truhlar's optimized meta-GGA of the ``M06'' suite of
        functionals. \cite{ZhaoTruhlar06_M06-L}
      \item \option{m11-l} : Truhlar's optimized range-separated local meta-GGA of the ``M11'' suite of
         functionals. \cite{PeveratiTruhlar11_M11-L}
      \item \option{revtpss} : Meta-GGA revTPSS functional of Ref. \cite{Perdew09_revTPSS, Perdew11_revTPSS}.
      \item \option{tpss} : Meta-GGA TPSS functional of Ref. \cite{Tao03_TPSS}
      \item \option{tpssloc} : Meta-GGA TPSSloc functional, thanks to E. Fabiano
         and F. Della Sala. L.A. Constantin, E. Fabiano, F.Della Sala,
         Ref. \cite{Constantin2012}.
      \item \option{scan} or \option{SCAN}: \emph{At present, there are
      substantial concerns about the accuracy of the potential of the canonical
      SCAN implementation. For self-consistent SCAN calculations, use the \texttt{dfauto} implementation of SCAN. See the \option{dfauto} option below.}
      % \item \option{scan} or \option{SCAN}: ``Strongly Constrained and
      %    Appropriately Normed Semilocal Density Functional,'' i.e., the
      %    SCAN meta-GGA functional by Sun, Ruzsinszky, and
      %    Perdew.\cite{Sun2015} \\
      %    \emph{Note: This functionality is very new at the time of the
      %    FHI-aims 2016 release in March, 2016. It has been tested, but
      %    please treat it with appropriate caution for a while.}
    \end{itemize}
  \item Hartree-Fock and hybrid functionals (including non-local exchange): 
     \emph{Please also see Secs. \ref{Sec:auxil} and \ref{Sec:periodic_hf} for related keywords and technical hints.}
    \begin{itemize}
      \item \option{b3lyp} : ``B3LYP'' hybrid functional as allegedly
        implemented in the Gaussian code (i.e., using the RPA version of the
        Vosk-Wilk-Nusair local-density approximation, see Refs. \cite{Vosko80,Scuseria05}
        for details). Note that this is therefore \emph{not} exactly
        the same B3LYP as originally described by Becke in 1993. 
      \item \option{hf} : Hartree-Fock exchange only.
      \item \option{hse03} : Hybrid functional as used in Heyd, Scuseria and
        Ernzerhof \cite{Heyd03,Heyd06}. In this functional, 25 \% of the exchange energy is
        split into a short-ranged, screened Hartree-Fock part, and a PBE
        GGA-like functional for the long-range part of exchange. The remaining
        75 \% exchange and full correlation energy are treated as in PBE. As
        clarified in Refs. \cite{Krukau06,Heyd06}, two different screening parameters
        were used in the short-range exchange part and long-range exchange
        part of the original HSE functional, respectively: \\
        Screened Hartree-Fock exchange: $\omega_\text{HF}=0.15/\sqrt{2}$ \\
        Screened PBE-like exchange: $\omega_\text{PBE}=0.15\times 2^{1/3}$ \\
        Following the notation of Ref. \cite{Krukau06}, the 'hse03' functional 
        in FHI-aims reproduces these original values exactly.
      \item \option{hse06} : Hybrid functional according to
        Heyd, Scuseria and Ernzerhof \cite{Heyd03}, following the naming convention
        suggested in Ref. \cite{Krukau06}. In this case, the
        additional option \option{value} is needed, representing the single
        real, positive screening parameter \option{omega} ($\omega$) as clarified
        in Ref. \cite{Krukau06}. In this functional, 25 \% of the
        exchange energy is split into a short-ranged, screened
        Hartree-Fock part, and a PBE GGA-like functional for the
        long-range part of exchange. The remaining
        75 \% exchange and full correlation energy are treated as in PBE. \\
        \emph{In the literature, the unit for $\omega$ is either {\AA}$^{-1}$ or (bohr radius)$^{-1}$, 
        depending on the code, authors, and their favorite convention. To avoid any confusion,
        a separate keyword \keyword{hse\_unit} must be specified in \texttt{control.in}, specifying
        either {\AA}$^{-1}$ ('A') or bohr$^{-1}$ ('b'). The code will no longer run without this
        explicit clarification. A correct calling syntax example is therefore:} \\
        \texttt{xc hse06 0.11} \\
        \texttt{hse\_unit bohr-1}\footnote{The \texttt{hse\_unit} flag reads only the first character.
                                           Thus this is equivalent to \texttt{hse\_unit b} (case insensitive).} \\
        or similar. \\
        A few comments on typical choices for $\omega$ in the earlier literature: \\
        The original value of 0.15 bohr$^{-1}$ by Heyd, Scuseria and Ernzerhoff 2003 \cite{Heyd03}
        was never true - see their 2006 erratum. In FHI-aims, the 'hse03' functional
        implements their actual choice. \\
        Krukau, Vydrov, Izmaylov and Scuseria 2006 \cite{Krukau06} clarify the distinction between
        'hse03' and 'hse06' (in addition to the Erratum mentioned above). Their
        conclusion is that omega=0.11 bohr$^{-1}$ is a reasonable choice.\\
        Vydrov, Heyd, Krukau and Scuseria in 2006 \cite{Vydrov06} appear to favor omega=0.25 bohr$^{-1}$,
        but with a mixing parameter (keyword \keyword{hybrid\_xc\_coeff}) of 0.5 for the short-range exchange.
        (The default for \keyword{hybrid\_xc\_coeff} in FHI-aims is 0.25, i.e., only a quarter of HF-like exchange.) \\
         You get the idea. As much as we would like to, we can not specify a single omega
         parameter for hse06 by default -- the choice is up to you. Apologies for the inconvenience.
      \item \option{pbe0} : PBE0 hybrid functional \cite{Adamo99}, mixing 75
        \% GGA exchange with 25 \% Hartree-Fock exchange.
      \item \option{pbesol0} : Hybrid functional in analogy to PBE0
        \cite{Adamo99}, except that the PBEsol \cite{Perdew08} GGA functionals
        are used, mixing 75 \% GGA exchange with 25 \% Hartree-Fock exchange.
      \item \option{lc\_wpbeh} : Range separated hybrid functional LC-$\omega$PBEh using 100 \% Hartree-Fock exchange in the 
long-range part
        and $\omega$PBE \cite{Vydrov06} in the short-range part. The full correlation energy is treated as in PBE. 
        The \keyword{hse\_unit} must be specified as in hse06!\\
        Syntax: \\
        \texttt{xc lc\_wpbeh $\omega$ $\alpha$}\\
        \begin{equation}
         E_{xc}^{\text{LC-}\omega\text{PBEh}} = \alpha E_{xx}^{\text{SR}} + (1-\alpha) E_{x_{\omega\text{PBE}}}^{\text{SR}} + 
         \left( \frac{1}{\epsilon} \right) E_{xx}^{\text{LR}} + \left( 1-\frac{1}{\epsilon} \right) 
E_{\omega\text{PBE}}^{\text{LR}} + E_{c_{\text{PBE}}}
        \end{equation}
        \begin{itemize}
         \item $\epsilon$ can be the dielectric constant. The default value is 1. One might change this parameter with the 
               keyword \keyword{lc\_dielectric\_constant}
         \item If $\alpha = 0$ the functional is also known as LC-$\omega$PBE \cite{Gallandi2015}
         \item $\alpha = 1$ would correspond to a PBE0 calcuation with 100 \% Hartree-Fock exchange
        \end{itemize}
    \end{itemize}
 \item Hybrid Meta-generalized gradient functionals (including non-local exchange): 
     \emph{Please also see Secs. \ref{Sec:auxil} and \ref{Sec:periodic_hf} for related keywords and technical hints.
     Currently the non-local exchange contribution is fixed in all implementations due to the parameterised nature
     of these density functionals.}
    \begin{itemize}
  \item \option{m06} : Truhlar's optimized hybrid meta-GGA of the ``M06'' suite of
    functionals; with 27\% exact exchange. \cite{ZhaoTruhlar06_M06_M06-2X}
  \item \option{m06-2x} : Truhlar's optimized hybrid meta-GGA of the ``M06'' suite of
    functionals, with double contribution (54\%) from the hartree-fock exact exchange. \cite{ZhaoTruhlar06_M06_M06-2X}
  \item \option{m06-hf} : Truhlar's optimized hybrid meta-GGA of the ``M06'' suite of
    functionals, with 100\% exact exchage contribution. \cite{ZhaoTruhlar06_M06-HF}
  \item \option{m08-hx} : Truhlar's optimized hybrid meta-GGA of the ``M08'' suite of
    functionals, with 52.23\% contribution from the hartree-fock exact exchange. \cite{ZhaoTruhlar08_M08-HX_M08-SO}
  \item \option{m08-so} : Truhlar's optimized hybrid meta-GGA of the ``M08'' suite of
    functionals, with 56.79\% contribution from the hartree-fock exact exchange. \cite{ZhaoTruhlar08_M08-HX_M08-SO}
  \item \option{m11} : Truhlar's optimized range-separated local meta-GGA of the ``M11'' suite of
  functionals \cite{PeveratiTruhlar11_M11}. The range-separation variable is also hardcoded in this implementation with
  $\omega = 0.25$ bohr$^{-1}$.
    \end{itemize}
  \item Alternative implementations of some XC functionals via the \verb+dfauto+ program \cite{Strange2001}. These implementations are generated automatically from Maple definitions that are located in \verb+xc_dfauto/+. The general syntax is \verb+xc dfauto <name>+ where \verb+<name>+ can be one of (case-insensitive):
  \begin{itemize}
    \item \verb+dfauto pw-lda|pbe|pbe0|tpss+ : This is practically identical to specifying directly \verb+xc <name>+, and essentially provides alternative implementations of those functionals for testing purposes.
    \item \verb+dfauto scan+ : This the meta-GGA functional SCAN \cite{Sun2015}.
    \item \verb+dfauto scan0+ : This the meta-GGA hybrid functional SCAN0 \cite{Hui2016}, which mixes SCAN with 25\% of exact exchange.
  \end{itemize}
  \item Double-hybrid functionals (including non-local exchange and correlation):
	  Double-hybrid functionals are emerging quickly in the last decade. ``\emph{double-hybrid}'' here
	  means that the exchange functional mixes LDA(and/or GGA) exchange with ``Hartree-Fock like exact exchange''.
	  Meanwhile, the correlation functional is composed of both conventional LDA(and/or GGA) correlation and
	  second-order perturbation energy. Doubly-hybrid functionals are ``\emph{semi-empirical}'', generally
	  including several empirical parameters determined by optimizing against one or several well-chosen databases.
	  Double-hybrid functionals show a remarkable improvement over conventional (hybrid-)GGAs in the description
	  of heats of formation, bond dissociation enthalpies, reaction barrier heights and weak interactions of the 
	  main group elements.
	  Doulbe-hybrid functionals have become new leading actors in the field of computational chemistry.
     \begin{itemize}
		 \item \option{xyg3} : Double-hybrid functional XYG3, containing 80.33\% Hartree-Fock exchange
			 and 32.11\% second-order perturbation energy \cite{Zhang2009}.
		 \item \option{xdh-pbe0} : Double-hybrid functional xDH-PBE0, containing 83.51\% Hartree-Fock exchange
			 and 52.42\% opposite-spin second-order perturbation correlation \cite{Zhang2012}.
     \end{itemize}
	  
  \item Some specific correlated methods: \emph{Only a subset. For
    many correlated methods that can be used as non-selfconsistent
    perturbative post-processing methods after an initial
    s.c.f. calculation, see the \keyword{total\_energy\_method}
    keyword. Most of these are not available for periodic
    geometries, or, if at all, in a very experimental state.}
    \begin{itemize}
      \item \option{mp2} : Self-consistent Hartree-Fock, followed by a
        second-order M{\o}ller-Plesset perturbative addition of the
        correlation energy. Note that the \keyword{frozen\_core} keyword can
        be used to specify if and which low-lying states should be excluded
        from the correlation energy. For spin-component scaled MP2
        \cite{Grimme03}, see keyword \keyword{scs\_mp2\_parameters}. \\[1.0ex]
        \emph{Note} that when mp2 is used for binding energy calculations,
        a \textbf{counterpoise correction} should always be performed to get reliable
        results, since the basis set superposition error (BSSE) for these correlation
        methods is significant. \emph{Note added in March 2016: A periodic
        implementation of MP2 is available but, at the time of writing,
        computationally extremely expensive. If you decide to use it,
        please keep in mind that the periodic version is included here
        as a matter of protocol but is not yet optimized to be fully
        usable in production calculations.}
      \item \option{screx} : \emph{experimental!} Self-consistent, screened
        Hartree-Fock exchange only. The Coulomb operator is screened as:
        \begin{equation}
          \frac{1}{r-r^\prime} \rightarrow \frac{1}{\varepsilon(r,r^\prime)}
          \cdot \frac{1}{r-r^\prime}
        \end{equation}
        $\varepsilon(r,r^\prime)$ is the non-local \emph{microscopic}
        dielectric function, obtained in the $\omega\rightarrow$0 frequency
        limit of the random-phase approximation (RPA). See Ref. \cite{Hedin65} for
        details.
      \item \option{cohsex} : \emph{experimental!} Self-consistent screened
        exchange plus Coulomb-hole (COH) correlation. See Ref. \cite{Hedin65} for
        details.
    \end{itemize}
  \item Method of non-local correlation using the ``van der Waals
    density functional'' (vdw-DF) as presented by Dion and coworkers
    in Ref.~\cite{Dion04}. Two options are available for the exchange
    part:  
    \begin{itemize}
      \item \option{pbe\_vdw} : the functional with pbe exchange
      \item \option{revpbe\_vdw} : the functional with revpbe exchange
    \end{itemize}
    Note that this keyword is \textbf{not} the correction due to
    Tkatchenko and Scheffler 2009 \cite{TS-vdw}. To activate the
    Tkatchenko-Scheffler correction instead, use the
    \keyword{vdw\_correction\_hirshfeld} keyword. The functional by
    Dion \emph{et al.} is a very different functional. As implemented
    here, it is also \emph{much} more expensive than the
    Tkatchenko-Scheffler correction. To use the functional by Dion
    \emph{et al.}, please review the numerical options described in
    Sec.~\ref{vdwdf-TKK}. 
\end{itemize}
\emph{Note} that our version of the Coulomb operator (which is the basis for
Hartree-Fock exchange also in hybrid functionals, as well as MP2 theory) is
based on an auxiliary basis in what is known as \emph{resolution of the
  identity} (Refs. \cite{Boys59,Alsenoy88,Vahtras93,Eichkorn95} and
others). While our default settings should be safe, you may
wish to consult Sec. \ref{Sec:auxil} for particulars regarding this auxiliary
basis.

\emph{Note also} that some different \emph{perturbative} exchange-correlation
treatments \emph{for post-processing} (after a self-consistent DFT or HF
calculation is complete) may be invoked using the tag
\keyword{total\_energy\_method}. Likewise, perturbative postprocessing for
single-quasiparticle energies through a self-energy formalism (e.g., $GW$) is
reached by specifying the \keyword{qpe\_calc} tag and its options.

Right now, the correlated beyond-hybrid and beyond-meta methods are not
implemented on top of the HSE03 or HSE06 functionals.

\subsubsection*{Subtags for \emph{species} tag in \texttt{control.in}:}

\subkeydefinition{species}{plus\_u}{control.in}
{
  \noindent
  Usage: \subkeyword{species}{plus\_u} \option{n} \option{l} \option{U} \\[1.0ex]
  Purpose: \emph{Experimental---only for DFT+U.} Adds a +U term to one
    specific shell of this species. \\[1.0ex] 
  \option{n} the (integer) radial quantum number of the selected shell. \\
  \option{l} is a character, specifying the angular momentum (
    \emph{s}, \emph{p}, \emph{d}, \emph{f}, ...) of the selected shell. \\
  \option{U} the value of the U parameter, specified in eV. \\
}

This implementation of DFT+U is based directly on the basis functions available within FHI-aims.
This option selects one specific \textit{atomic} shell of this species and adds the a rotationally
invariant term with the specified fixed prefactor U to the Hamiltonian. The implementation follows
the prescription in Ref.~\cite{Han06}, based on the \textit{dual} occupation numbers. The double
counting term is handled through the mixed term proposed by Petukhov
(see \keyword{plus\_u\_petukhov\_mixing}).

\chapter{XML output}
\label{appendix_xml_output}

The FHI-aims can print results of the calculations in an XML format. This 
functionality is provided by the \verb`xml_write` module. The module is quite 
general and capable of printing any XML, but also has convenient functions for 
printing scalars, vectors, matrices and $n$-dimensional arrays. The formatting 
of the data in the XML elements is both human-readable and machine-readable. The 
obvious advantage is that such an output from the code is extremely easy and 
fast to parse.

There are two ways how the XML module can be used. There is a global XML file 
which can be turned on with \verb`xml_file <filename>` in \verb`control.in`.  
Printing to this file is as simple as loading the module and calling one of its 
routines. Note however that if the block of code that prints to this global file
calls another code also printing to the global file, the XML output of these two
codes might get intertwined, resulting in a valid but nonsensical XML file.

A second way how to use the XML module is to create a new ``local'' XML file
with the \verb`open_xml_file` routine and use that one by calling the module 
printing routines with the optional argument \verb`file=<xml_file_instance>`.

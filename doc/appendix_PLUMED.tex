%\newcommand{\doublekeydefinition}[3]
% {{\textbf{Tag: \texttt{#1}}{ \color{filename_color} (#2)}}
% #3
% 
% }

\section{Plugin for free-energy calculations with molecular dynamics: \texttt{PLUMED}}
\label{appendix_PLUMED}


Molecular dynamics based free-energy calculations can be performed
with the aid of the external plugin \texttt{PLUMED}.\\
Methods included are metadynamics \cite{mtd}, well-tempered metadynamics \cite{wmtd}, umbrella sampling \cite{usa1,usa2,usa3}, Jarzynski-equation based steered molecular dynamics \cite{smd1,smd2}.
A large and nearly exhaustive set of collective variable (CV) is accessible through a simple input script.

\texttt{PLUMED} is a free package that, after registration, can be downloaded from 
\url{http://merlino.mi.infn.it/~plumed/PLUMED/Home.html}
Currently, a copy of the \texttt{PLUMED} library is
kept in the \emph{external} directory of the FHI-aims source code, and
must be compiled separately using the makefile Makefile.meta (see section
\ref{Sec:Makefiles}). In the future a patch for modifying FHI-aims in order to compile it with PLUMED will be available on the PLUMED webpage.\\

\subsection{Usage}

The actual use of the plugin is switched on by this single line in \texttt{control.in}:
\begin{verbatim}
 plumed .true.
\end{verbatim}
With \texttt{plumed .false.} (default) or nothing, the code would behave exactly as compiled without
this plugin. It is implied that some MD scheme must be used in \texttt{control.in}, in order to see PLUMED acting. What PLUMED does, in facts, is to modify the molecular dynamics forces according to the selected scheme.\\
All the specific controls of the free energy calculation are contained in the file \texttt{plumed.dat} (which must be in the working directory, together with \texttt{control.in} and \texttt{geometry.in}, if \texttt{plumed .true.} is set).
For all the details on \texttt{plumed.dat}, we defer to \texttt{PLUMED} manual which can be found on the project website.\\
Here we report a minimal example for metadynamics:
\begin{verbatim}
PRINT W_STRIDE 10
DISTANCE LIST 1 <g1> SIGMA 0.35
g1->
2 3 4
g1<-
HILLS HEIGHT 0.003 W_STRIDE 10
ENDMETA
\end{verbatim}
This script would make \texttt{PLUMED} deposit Gaussians (\texttt{HILLS}) of \texttt{HEIGHT} $0.003$ hartree, every \texttt{W\_STRIDE} timesteps.
The (only) CV that will be biased by metadynamics is a distance between atom `1' and the center of mass of atoms `2', `3', and `4'. The number labelling the atoms follows their order of appearance in \texttt{geometry.in}. The results will be printed (see below) every \texttt{PRINT W\_STRIDE} time steps.
The width of the Gaussian for the distance CV is specified by \texttt{SIGMA}

A note on the units: the units in \texttt{plumed.dat} and in the output(s) are the internal ones in FHI-aims, i.e. energies in hartree, distances in bohr, forces in hartree/bohr.

When using \texttt{PLUMED}, some extra output files are created. 
In \texttt{log.dat} the specifics of the run are given. \texttt{COLVAR} contains the trajectory of
the selected CVs. 
Notably, if no biasing method is selected, but one or more CVs are defined in \texttt{plumed.dat}, PLUMED prints nonetheless the trajectory of those CVs in \texttt{COLVAR} (one can also explicitly switch off the biasing of \textit{some} CVs via the \texttt{NOHILLS} directive).\\
In case metadynamics is used, then also \texttt{HILLS} is generated, which contains the informations for reconstructing the free energy profile. This is done with the postprocessing tool, ``sum\_hills'', which is given with the distribution. \\
For umbrella sampling a powerful tool for reconstructing the free-energy from
\texttt{COLVAR}, can be downloaded from: \url{http://membrane.urmc.rochester.edu/Software/WHAM/WHAM.html}.





% \option{NEB.pl}
% \keyword{input\_template}. 
% 
% \keydefinition{aims\_control\_file\_template}{control\_NEB.in}
% {\begin{itemize}
%   \item Usage: \keyword{aims\_control\_file\_template} \option{filename}
%   \item Location of the AIMS template file to be used for the TS search.
%   \end{itemize}}
% 
% \keydefinition{input\_template}{control\_NEB.in}
% {\begin{itemize}
%     \item Usage: \keyword{input\_template} \option{filename}
%     \item The location of the initial guess for the TS-search. The file name contains 
%       the number of the iteration, followed by \option{.xyz}, while 
%       \keyword{input\_template} should be without those two points. 
%     \item Example: An initial guess called \option{space\_dog\_0.xyz} 
%       would be described with \\
%       \option{input\_template space\_dog\_}
%     \item This file has to be a multi-frame xyz file with the following format:\\
%       \option{<n\_atoms>} \\
%       \option{start} \\
%       \option{<species1> <xstart> <ystart> <zstart> }\\
%       $\ldots$\\
%       \option{<n\_atoms>}\\
%       \option{image 1}\\
%       \option{<species1> <x1> <y1> <z1> }\\
%       $\ldots$\\
%       \option{<n\_atoms>}\\
%       \option{image <n\_images>}\\
%       \option{<species1> <x1> <y1> <z1> }\\
%       $\ldots$\\
%       \option{<n\_atoms>}\\
%       \option{END}\\
%       \option{<species1> <x1> <y1> <z1> }\\
%       $\ldots$\\      
%   \end{itemize}}
% 
% \doublekeydefinition{max\_atomic\_move}{control\_NEB.in}
% {\begin{itemize}
%     \item Usage: \option{max\_atomic\_move} \option{value}
%     \item The maximally allowed displacement during a single NEB step.
%     \item This value should be relatively small (default is 0.1\AA) as the
%       algorithm might literally tear apart some of the images if the 
%       force constants are not set ideally and the transition path is not 
%       extremely close to the actual path. 
%     \item Same as keyword \keyword{max\_atomic\_move} in AIMS
%   \end{itemize}}
% 
